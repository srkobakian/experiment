%% template.tex
%% from
%% bare_conf.tex
%% V1.4b
%% 2015/08/26
%% by Michael Shell
%% See:
%% http://www.michaelshell.org/
%% for current contact information.
%%
%% This is a skeleton file demonstrating the use of IEEEtran.cls
%% (requires IEEEtran.cls version 1.8b or later) with an IEEE
%% conference paper.
%%
%% Support sites:
%% http://www.michaelshell.org/tex/ieeetran/
%% http://www.ctan.org/pkg/ieeetran
%% and
%% http://www.ieee.org/

%%*************************************************************************
%% Legal Notice:
%% This code is offered as-is without any warranty either expressed or
%% implied; without even the implied warranty of MERCHANTABILITY or
%% FITNESS FOR A PARTICULAR PURPOSE!
%% User assumes all risk.
%% In no event shall the IEEE or any contributor to this code be liable for
%% any damages or losses, including, but not limited to, incidental,
%% consequential, or any other damages, resulting from the use or misuse
%% of any information contained here.
%%
%% All comments are the opinions of their respective authors and are not
%% necessarily endorsed by the IEEE.
%%
%% This work is distributed under the LaTeX Project Public License (LPPL)
%% ( http://www.latex-project.org/ ) version 1.3, and may be freely used,
%% distributed and modified. A copy of the LPPL, version 1.3, is included
%% in the base LaTeX documentation of all distributions of LaTeX released
%% 2003/12/01 or later.
%% Retain all contribution notices and credits.
%% ** Modified files should be clearly indicated as such, including  **
%% ** renaming them and changing author support contact information. **
%%*************************************************************************


% *** Authors should verify (and, if needed, correct) their LaTeX system  ***
% *** with the testflow diagnostic prior to trusting their LaTeX platform ***
% *** with production work. The IEEE's font choices and paper sizes can   ***
% *** trigger bugs that do not appear when using other class files.       ***                          ***
% The testflow support page is at:
% http://www.michaelshell.org/tex/testflow/

\documentclass[conference,final,]{IEEEtran}
% Some Computer Society conferences also require the compsoc mode option,
% but others use the standard conference format.
%
% If IEEEtran.cls has not been installed into the LaTeX system files,
% manually specify the path to it like:
% \documentclass[conference]{../sty/IEEEtran}





% Some very useful LaTeX packages include:
% (uncomment the ones you want to load)


% *** MISC UTILITY PACKAGES ***
%
%\usepackage{ifpdf}
% Heiko Oberdiek's ifpdf.sty is very useful if you need conditional
% compilation based on whether the output is pdf or dvi.
% usage:
% \ifpdf
%   % pdf code
% \else
%   % dvi code
% \fi
% The latest version of ifpdf.sty can be obtained from:
% http://www.ctan.org/pkg/ifpdf
% Also, note that IEEEtran.cls V1.7 and later provides a builtin
% \ifCLASSINFOpdf conditional that works the same way.
% When switching from latex to pdflatex and vice-versa, the compiler may
% have to be run twice to clear warning/error messages.






% *** CITATION PACKAGES ***
%
%\usepackage{cite}
% cite.sty was written by Donald Arseneau
% V1.6 and later of IEEEtran pre-defines the format of the cite.sty package
% \cite{} output to follow that of the IEEE. Loading the cite package will
% result in citation numbers being automatically sorted and properly
% "compressed/ranged". e.g., [1], [9], [2], [7], [5], [6] without using
% cite.sty will become [1], [2], [5]--[7], [9] using cite.sty. cite.sty's
% \cite will automatically add leading space, if needed. Use cite.sty's
% noadjust option (cite.sty V3.8 and later) if you want to turn this off
% such as if a citation ever needs to be enclosed in parenthesis.
% cite.sty is already installed on most LaTeX systems. Be sure and use
% version 5.0 (2009-03-20) and later if using hyperref.sty.
% The latest version can be obtained at:
% http://www.ctan.org/pkg/cite
% The documentation is contained in the cite.sty file itself.






% *** GRAPHICS RELATED PACKAGES ***
%
\ifCLASSINFOpdf
  % \usepackage[pdftex]{graphicx}
  % declare the path(s) where your graphic files are
  % \graphicspath{{../pdf/}{../jpeg/}}
  % and their extensions so you won't have to specify these with
  % every instance of \includegraphics
  % \DeclareGraphicsExtensions{.pdf,.jpeg,.png}
\else
  % or other class option (dvipsone, dvipdf, if not using dvips). graphicx
  % will default to the driver specified in the system graphics.cfg if no
  % driver is specified.
  % \usepackage[dvips]{graphicx}
  % declare the path(s) where your graphic files are
  % \graphicspath{{../eps/}}
  % and their extensions so you won't have to specify these with
  % every instance of \includegraphics
  % \DeclareGraphicsExtensions{.eps}
\fi
% graphicx was written by David Carlisle and Sebastian Rahtz. It is
% required if you want graphics, photos, etc. graphicx.sty is already
% installed on most LaTeX systems. The latest version and documentation
% can be obtained at:
% http://www.ctan.org/pkg/graphicx
% Another good source of documentation is "Using Imported Graphics in
% LaTeX2e" by Keith Reckdahl which can be found at:
% http://www.ctan.org/pkg/epslatex
%
% latex, and pdflatex in dvi mode, support graphics in encapsulated
% postscript (.eps) format. pdflatex in pdf mode supports graphics
% in .pdf, .jpeg, .png and .mps (metapost) formats. Users should ensure
% that all non-photo figures use a vector format (.eps, .pdf, .mps) and
% not a bitmapped formats (.jpeg, .png). The IEEE frowns on bitmapped formats
% which can result in "jaggedy"/blurry rendering of lines and letters as
% well as large increases in file sizes.
%
% You can find documentation about the pdfTeX application at:
% http://www.tug.org/applications/pdftex





% *** MATH PACKAGES ***
%
%\usepackage{amsmath}
% A popular package from the American Mathematical Society that provides
% many useful and powerful commands for dealing with mathematics.
%
% Note that the amsmath package sets \interdisplaylinepenalty to 10000
% thus preventing page breaks from occurring within multiline equations. Use:
%\interdisplaylinepenalty=2500
% after loading amsmath to restore such page breaks as IEEEtran.cls normally
% does. amsmath.sty is already installed on most LaTeX systems. The latest
% version and documentation can be obtained at:
% http://www.ctan.org/pkg/amsmath





% *** SPECIALIZED LIST PACKAGES ***
%
%\usepackage{algorithmic}
% algorithmic.sty was written by Peter Williams and Rogerio Brito.
% This package provides an algorithmic environment fo describing algorithms.
% You can use the algorithmic environment in-text or within a figure
% environment to provide for a floating algorithm. Do NOT use the algorithm
% floating environment provided by algorithm.sty (by the same authors) or
% algorithm2e.sty (by Christophe Fiorio) as the IEEE does not use dedicated
% algorithm float types and packages that provide these will not provide
% correct IEEE style captions. The latest version and documentation of
% algorithmic.sty can be obtained at:
% http://www.ctan.org/pkg/algorithms
% Also of interest may be the (relatively newer and more customizable)
% algorithmicx.sty package by Szasz Janos:
% http://www.ctan.org/pkg/algorithmicx




% *** ALIGNMENT PACKAGES ***
%
%\usepackage{array}
% Frank Mittelbach's and David Carlisle's array.sty patches and improves
% the standard LaTeX2e array and tabular environments to provide better
% appearance and additional user controls. As the default LaTeX2e table
% generation code is lacking to the point of almost being broken with
% respect to the quality of the end results, all users are strongly
% advised to use an enhanced (at the very least that provided by array.sty)
% set of table tools. array.sty is already installed on most systems. The
% latest version and documentation can be obtained at:
% http://www.ctan.org/pkg/array


% IEEEtran contains the IEEEeqnarray family of commands that can be used to
% generate multiline equations as well as matrices, tables, etc., of high
% quality.




% *** SUBFIGURE PACKAGES ***
%\ifCLASSOPTIONcompsoc
%  \usepackage[caption=false,font=normalsize,labelfont=sf,textfont=sf]{subfig}
%\else
%  \usepackage[caption=false,font=footnotesize]{subfig}
%\fi
% subfig.sty, written by Steven Douglas Cochran, is the modern replacement
% for subfigure.sty, the latter of which is no longer maintained and is
% incompatible with some LaTeX packages including fixltx2e. However,
% subfig.sty requires and automatically loads Axel Sommerfeldt's caption.sty
% which will override IEEEtran.cls' handling of captions and this will result
% in non-IEEE style figure/table captions. To prevent this problem, be sure
% and invoke subfig.sty's "caption=false" package option (available since
% subfig.sty version 1.3, 2005/06/28) as this is will preserve IEEEtran.cls
% handling of captions.
% Note that the Computer Society format requires a larger sans serif font
% than the serif footnote size font used in traditional IEEE formatting
% and thus the need to invoke different subfig.sty package options depending
% on whether compsoc mode has been enabled.
%
% The latest version and documentation of subfig.sty can be obtained at:
% http://www.ctan.org/pkg/subfig




% *** FLOAT PACKAGES ***
%

%\usepackage{fixltx2e}
% fixltx2e, the successor to the earlier fix2col.sty, was written by
% Frank Mittelbach and David Carlisle. This package corrects a few problems
% in the LaTeX2e kernel, the most notable of which is that in current
% LaTeX2e releases, the ordering of single and double column floats is not
% guaranteed to be preserved. Thus, an unpatched LaTeX2e can allow a
% single column figure to be placed prior to an earlier double column
% figure.
% Be aware that LaTeX2e kernels dated 2015 and later have fixltx2e.sty's
% corrections already built into the system in which case a warning will
% be issued if an attempt is made to load fixltx2e.sty as it is no longer
% needed.
% The latest version and documentation can be found at:
% http://www.ctan.org/pkg/fixltx2e


%\usepackage{stfloats}
% stfloats.sty was written by Sigitas Tolusis. This package gives LaTeX2e
% the ability to do double column floats at the bottom of the page as well
% as the top. (e.g., "\begin{figure*}[!b]" is not normally possible in
% LaTeX2e). It also provides a command:
%\fnbelowfloat
% to enable the placement of footnotes below bottom floats (the standard
% LaTeX2e kernel puts them above bottom floats). This is an invasive package
% which rewrites many portions of the LaTeX2e float routines. It may not work
% with other packages that modify the LaTeX2e float routines. The latest
% version and documentation can be obtained at:
% http://www.ctan.org/pkg/stfloats
% Do not use the stfloats baselinefloat ability as the IEEE does not allow
% \baselineskip to stretch. Authors submitting work to the IEEE should note
% that the IEEE rarely uses double column equations and that authors should try
% to avoid such use. Do not be tempted to use the cuted.sty or midfloat.sty
% packages (also by Sigitas Tolusis) as the IEEE does not format its papers in
% such ways.
% Do not attempt to use stfloats with fixltx2e as they are incompatible.
% Instead, use Morten Hogholm'a dblfloatfix which combines the features
% of both fixltx2e and stfloats:
%
% \usepackage{dblfloatfix}
% The latest version can be found at:
% http://www.ctan.org/pkg/dblfloatfix




% *** PDF, URL AND HYPERLINK PACKAGES ***
%
%\usepackage{url}
% url.sty was written by Donald Arseneau. It provides better support for
% handling and breaking URLs. url.sty is already installed on most LaTeX
% systems. The latest version and documentation can be obtained at:
% http://www.ctan.org/pkg/url
% Basically, \url{my_url_here}.




% *** Do not adjust lengths that control margins, column widths, etc. ***
% *** Do not use packages that alter fonts (such as pslatex).         ***
% There should be no need to do such things with IEEEtran.cls V1.6 and later.
% (Unless specifically asked to do so by the journal or conference you plan
% to submit to, of course. )



%% BEGIN MY ADDITIONS %%


\usepackage{longtable,booktabs}
\usepackage{graphicx}
% We will generate all images so they have a width \maxwidth. This means
% that they will get their normal width if they fit onto the page, but
% are scaled down if they would overflow the margins.
\makeatletter
\def\maxwidth{\ifdim\Gin@nat@width>\linewidth\linewidth
\else\Gin@nat@width\fi}
\makeatother
\let\Oldincludegraphics\includegraphics
\renewcommand{\includegraphics}[1]{\Oldincludegraphics[width=\maxwidth]{#1}}

\usepackage[unicode=true]{hyperref}

\hypersetup{
            pdftitle={Comparing the Effectiveness of the Choropleth Map with a Hexagon Tile Map for Communicating Cancer Statistics},
            pdfkeywords={statistics, visual inference, geospatial, population},
            pdfborder={0 0 0},
            breaklinks=true}
\urlstyle{same}  % don't use monospace font for urls

% Pandoc toggle for numbering sections (defaults to be off)
\setcounter{secnumdepth}{5}

% Pandoc syntax highlighting


% Pandoc header
\usepackage{graphicx}
\usepackage{booktabs}
\usepackage{longtable}
\usepackage{array}
\usepackage{multirow}
\usepackage{wrapfig}
\usepackage{float}
\usepackage{colortbl}
\usepackage{pdflscape}
\usepackage{tabu}
\usepackage{threeparttable}
\usepackage{threeparttablex}
\usepackage[normalem]{ulem}
\usepackage{makecell}
\usepackage{xcolor}

\providecommand{\tightlist}{%
  \setlength{\itemsep}{0pt}\setlength{\parskip}{0pt}}

%% END MY ADDITIONS %%


\hyphenation{op-tical net-works semi-conduc-tor}

\begin{document}
%
% paper title
% Titles are generally capitalized except for words such as a, an, and, as,
% at, but, by, for, in, nor, of, on, or, the, to and up, which are usually
% not capitalized unless they are the first or last word of the title.
% Linebreaks \\ can be used within to get better formatting as desired.
% Do not put math or special symbols in the title.
\title{Comparing the Effectiveness of the Choropleth Map with a Hexagon Tile Map for Communicating Cancer Statistics}

% author names and affiliations
% use a multiple column layout for up to three different
% affiliations

\author{

%% ---- classic IEEETrans wide authors' list ----------------
 % -- end affiliation.wide
%% ----------------------------------------------------------



%% ---- classic IEEETrans one column per institution --------
 %% -- end if/affiliation.institution-columnar
%% ----------------------------------------------------------





%% ---- one column per author, classic/default IEEETrans ----
 % -- beg affiliation.author-columnar
  %% -- beg for/affiliation.institution.author
\IEEEauthorblockN{
Stephanie Kobakian
}
\IEEEauthorblockA{Queensland University of Technology\\
Science and Engineering Faculty\\
Brisbane, Australia
\\stephanie.kobakian@qut.edu.au
}
 %% -- end for/affiliation.institution.author
\and
  %% -- beg for/affiliation.institution.author
\IEEEauthorblockN{
Dianne Cook
}
\IEEEauthorblockA{Monash University\\
Econometrics and Business Statistics Faculty\\
Melbourne, Australia
\\dicook@monash.edu
}
 %% -- end for/affiliation.institution.author
 %% -- end for/affiliation.institution
 %% -- end if/affiliation.institution-columnar
%% ----------------------------------------------------------

}

% conference papers do not typically use \thanks and this command
% is locked out in conference mode. If really needed, such as for
% the acknowledgment of grants, issue a \IEEEoverridecommandlockouts
% after \documentclass

% for over three affiliations, or if they all won't fit within the width
% of the page, use this alternative format:
%
%\author{\IEEEauthorblockN{Michael Shell\IEEEauthorrefmark{1},
%Homer Simpson\IEEEauthorrefmark{2},
%James Kirk\IEEEauthorrefmark{3},
%Montgomery Scott\IEEEauthorrefmark{3} and
%Eldon Tyrell\IEEEauthorrefmark{4}}
%\IEEEauthorblockA{\IEEEauthorrefmark{1}School of Electrical and Computer Engineering\\
%Georgia Institute of Technology,
%Atlanta, Georgia 30332--0250\\ Email: see http://www.michaelshell.org/contact.html}
%\IEEEauthorblockA{\IEEEauthorrefmark{2}Twentieth Century Fox, Springfield, USA\\
%Email: homer@thesimpsons.com}
%\IEEEauthorblockA{\IEEEauthorrefmark{3}Starfleet Academy, San Francisco, California 96678-2391\\
%Telephone: (800) 555--1212, Fax: (888) 555--1212}
%\IEEEauthorblockA{\IEEEauthorrefmark{4}Tyrell Inc., 123 Replicant Street, Los Angeles, California 90210--4321}}




% use for special paper notices
%\IEEEspecialpapernotice{(Invited Paper)}




% make the title area
\maketitle

% As a general rule, do not put math, special symbols or citations
% in the abstract
\begin{abstract}
The choropleth map display is commonly used for communicating spatial distributions across geographic areas. However, when choropleths are used, the sizes of areas can lead to the misinterpretation of the distribution. The visualization method used to present geospatial data will influence the understanding of the distribution derived by map users. Choosing an effective alternative display could positively influence the communication of the spatial distribution. The hexagon tile map is presented as an alternative display for visualizing population related distributions effectively. Visual inference is used to measure the power of design, and the choropleth is used as a comparison. The hexagon tile map display is also tested using a distribution that is directly related to the geography, with values monotonically decreasing from the North-West to South-East areas of Australia. This study finds the single map in a hexagon tile map lineup that contains a population related distribution is detected with greater probability than the same data displayed in a choropleth map. These findings should encourage map creators to implement alternative displays and consider a hexagon tile map when presenting spatial distributions of heterogeneous areas.
\end{abstract}

% keywords
\begin{IEEEkeywords}
statistics; visual inference; geospatial; population
\end{IEEEkeywords}

% use for special paper notices



% make the title area
\maketitle

% no keywords

% For peer review papers, you can put extra information on the cover
% page as needed:
% \ifCLASSOPTIONpeerreview
% \begin{center} \bfseries EDICS Category: 3-BBND \end{center}
% \fi
%
% For peerreview papers, this IEEEtran command inserts a page break and
% creates the second title. It will be ignored for other modes.
\IEEEpeerreviewmaketitle


\hypertarget{introduction}{%
\section{Introduction}\label{introduction}}

This study compares the effectiveness of a new type of display, a hexagon tile map, against the standard, a choropleth map, for communicating information about disease statistics. The choropleth map is the traditional approach for visualizing aggregated statistics across administrative boundaries. A hexagon tile map forgoes the familiar boundaries, in favor of representing each geographic unit as an equally sized hexagon, placed approximately in the correct spatial location. The hexagon tile map builds on existing displays, such as the cartogram, and tesselated hexagon displays. It differs in that it relaxes the requirement to have connected hexagons, and allows sparsely located hexagons. The algorithm to construct a hexagon tile map is available in the R package sugarbag {[}\protect\hyperlink{ref-sugarbag}{1}{]}. This type of display may be useful for other countries, and other purposes.

The hexagon tile map was designed for Australia, motivated by a need to display spatial statistics for a new Australian Cancer Atlas. None of the existing approaches for creating cartograms or hexagon tiling perform well for the Australian landscape, which has vast open spaces and concentrations of population in small regions clustered on the coastlines.

The Australian Cancer Atlas {[}\protect\hyperlink{ref-atlas}{2}{]} is an online interactive web tool created to explore the burden of cancer on Australian communities. There are many cancer types to be explored individually or aggregated. The Australian Cancer Atlas allows users to explore the patterns in the distributions of cancer statistics over the geographic space of Australia. It uses a choropleth map display and diverging color scheme to draw attention to relationships between neighboring areas. The hexagon tile map may be a useful alternative display for the atlas.

The experiment was conducted using the lineup protocol, a visual inference procedure {[}\protect\hyperlink{ref-GIIV}{3}{]}, to objectively test the effectiveness of the two displays.

The paper is organised as follows. The next section discusses the background of geographic data display and visual inference procedures. The \protect\hyperlink{methodology}{Methodology} Section describes the methods for conducting the experiment and analysing the results. The results are summarized in Section \protect\hyperlink{results}{Results}.

\hypertarget{background}{%
\section{Background}\label{background}}

\hypertarget{population-focussed-displays}{%
\subsection{Population focussed displays}\label{population-focussed-displays}}

Spatial visualizations communicate the distribution of statistics over geographic space. The most common display for spatial data is the choropleth map. It is used to present aggregated statistics for geographic units such as the population. Creating a choropleth map involves drawing the administrative boundaries and filling them with color to communicate the value of the statistic. However, using a choropleth map base introduces biases when considering population related distributions {[}\protect\hyperlink{ref-CBATCC}{4}{]}.
Many sets of Australian statistical areas have a large disparity between the smallest and largest units.
The rural communities in Australia operate on a much larger geographic space than small inner-city communities. When a spatial distribution relates to the size of the areas, or the population density, the size of the regions can result in erroneous conclusions about the state of the statistic over the entire population, when presenting the statistics on a geographic map base. This occurs as large regions filled with a consistent color or pattern can draw the attention of map readers, and small regions are not paid equal attention {[}\protect\hyperlink{ref-CTTMB}{5}{]}. An alternative display can be used to honestly communicate a spatial distribution for a set of heterogeneous areas {[}\protect\hyperlink{ref-NISCC}{6}{]}.
This work aims to show that a hexagon tile map display is a viable alternative to the geographic map base for presenting population statistics.

A choropleth map is not the only display that can be used for presenting geospatial data.
Alternative maps include various cartograms and tessellated tile maps. They allow other variables to be incorporated in the display to highlight the statistical values of various geographic areas.

A cartogram transforms the geographic map base. The transformation begins with a choropleth map and the values of a statistic such as population for each geographic unit. There are several algorithms for transformation {[}\protect\hyperlink{ref-ACTUC}{7}{]}, {[}\protect\hyperlink{ref-CBATCC}{4}{]}; they involve shifting the boundaries of geographic units, using the value of the statistic to increase or decrease the area taken by the geographic unit on the map.
The changes to the boundaries result in cartograms that accurately communicate population by map area for each of the geographic units. The transformation will keep boundaries of neighboring areas connected, but the result is unfamiliar shapes for each geographic unit.

Alternative displays make the trade off between familiar shapes and representation of variables using the area of geographic units. The non-contiguous cartogram method {[}\protect\hyperlink{ref-NAC}{8}{]} keeps the shapes of geographic units intact and changes the size of the shape to communicate the values of statistics. This method disconnects areas when shrinking boundaries and breaks the spatial relationship between areas. However, the empty space created emphasizes the comparative density of the statistic, the difference between the statistic and the amount of land for each geographic unit.
The Dorling cartogram {[}\protect\hyperlink{ref-ACTUC}{7}{]} is another method that does not keep the boundaries of geographic units. Each unit becomes one circle on the cartogram display; it sizes the circles according to the value of the statistic, such as population. The neighbor relationships can be illustrated by connecting the boundaries of the circles. A similar approach was pioneered by Raisz {[}\protect\hyperlink{ref-RSCW}{9}{]}, using rectangles that tile to align borders of neighbors {[}\protect\hyperlink{ref-CDWCS}{10}{]}.
These cartogram methods all allow two variables to be shown in the single display. These methods are discussed further in XXX cite review paper.

\begin{figure}
\centering
\includegraphics{paper_files/figure-latex/liver-1.pdf}
\caption{\label{fig:liver}The smoothed average of liver cancer diagnoses for Australian males. The divering color scheme shows dark blue areas with much lower than average diagnoses, yellow areas with diagnoses around the Australian average, orange with above average and red shows diagnoses much higher than average.}
\end{figure}

Australia is an extreme case of heterogeneous geographic units, with the large differences between the smallest and largest geographic units across various granularities.
To communicate information regarding the small inner-city geographic units of Australian cities, an alternative display must be employed. This will emphasize the value of the statistic for areas with a high population. Fig. \ref{fig:liver} shows the distribution of the smoothed average amount of liver cancer diagnoses for Australian Males from 2005 to 2014.
These inner-city areas are not visible on the choropleth map (a). The hexagon tile map (b) shows orange hexagons for the inner-city SA2s with higher than average levels of diagnoses in the capital cities of Melbourne and Sydney. The northern Queensland and Northern Territory SA2s with higher than average rates are still visible but use less map space on the hexagon tile map. There are still many light blue areas in both displays.

\hypertarget{visual-inference}{%
\subsection{Visual Inference}\label{visual-inference}}

We suggest the hexagon tile map as an alternative display. To test the effectiveness of these displays, there is a formal testing framework. Visual inference considers the communication of data or data summaries through visualizations. It considers these visualizations to be visual statistics, with observable features.

Classical statistical inference involves hypothesis testing, rejecting a null hypothesis in favor of an alternative. This approach requires data with the appropriate distributions and assumptions.
The lineup protocol was formalized as a test for the effectiveness of a visualization. Rather than classical statistical tests, it tasks human participants with finding the display that shows structure between the variables. If they do not detect the structure, then it is not significantly different from the null data.

Visual inference uses the general hypothesis structure:

\begin{itemize}
\tightlist
\item
  Null hypothesis: There is \bf{no} relationship between the variables
\item
  Alternative hypothesis: There is a relationship between the variables
\end{itemize}

To test these hypotheses, the line up protocol randomly places a ``guilty'' data visualization in a lineup of ``innocents''. Where the guilty data set contains structure, and the innocents data sets do not.
In a grid of visualizations, an observer is asked to pick the display that is most different, if they select the data set containing structure, they have identified the guilty hidden within the innocents {[}\protect\hyperlink{ref-GTPCCD}{11}{]}.
The participants identify the guilty data as different from the innocent data with probability \(1/m\), where \(m\) is the number of null plots plus 1 to account for the guilty data set. When the guilty data set is chosen we reject the null hypothesis that it was innocent is rejected with a \(1/m\) chance or type I error of being wrong.

The lineup protocol can be used in a variety of testing scenarios for visualizations. {[}\protect\hyperlink{ref-GIIV}{3}{]} suggest the choropleth map for testing spatial structure in a data set. The lineup protocol allows for flexibility in the definition of null data. In the case of spatial visualizations, it is likely that neighbors are related. To account for this, null data sets can be generated by sampling from a known model that accounts for spatial covariance.

To contrast the effectiveness of two displays, we can produce null data plots with a hidden real data plot using the different plot designs.
We can compare the time taken by participants to evaluate the same data in the different displays. We can also contrast the accuracy achieved by participants in detecting the real data plot.

\hypertarget{methodology}{%
\section{Methodology}\label{methodology}}

This study aimed to answer two key questions around the presentation of spatial distributions:

\begin{enumerate}
\def\labelenumi{\arabic{enumi}.}
\item
  Are spatial disease trends that impact highly populated small areas detected with higher accuracy, when viewed in a hexagon tile map display?
\item
  Are people faster in detecting spatial disease trends that impact highly populated small areas when using a hexagon tile map display?
\end{enumerate}

Additional considerations when completing this experimental task included exploration of the difficulty experienced by participants and the certainty they felt about their decision.

The mean of the detection rate for choropleth map, denoted as \(\mu_C\), and the hexagon tile maps, \(\mu_H\) will be contrasted. This leads to the following one sided hypothesis:

\(H_0\) : \(\mu_H\) = \(\mu_C\)
\(H_a\) : \(\mu_H\) \textgreater{} \(\mu_C\)

The detection rate \(\hat\pi\) is calculated as the amount of people that made the choice of plot that contained the real data, out of the participants who saw data plot in the lineup of the null data plots.

\hypertarget{participants}{%
\subsection{Participants}\label{participants}}

We recruited participants for the survey style task to test the effectiveness of the hexagon tile map display.
The lineup protocol expects that the participants are uninvolved judges that had no prior knowledge of the data to avoid discrimination or advantages.
The Figure-Eight crowd source platform was used to advertise the survey to potential participants that had achieved level 2 or level 3 on the Figure-Eight Platform.
The participants were chosen as they have experience with survey tasks that involve evaluating images. The participants were at least 18 years old, as it is a requirement of the platform.

Participants were able to choose to participate by selecting this task from the list of tasks available to them.
Participants were allocated to either group A or group B when they proceeded to the survey web application, hosted externally from the Figure-Eight website.
There were 95 participants involved in the study. All of these participants read the introductory materials and viewed example questions before proceeding to the survey. Each participant was trained using three test displays orienting them to the evaluation task.
All participants who volunteered to take part were compensated for their time via the payment system of Figure-Eight.

\hypertarget{variables}{%
\subsection{Variables}\label{variables}}

The variables changed between groups were the type of plot shown and the trend model.

A combination of map type and spatial trend model created the twenty-four lineup displays. We split this set of displays into a collection of twelve displays for Group A and twelve displays for Group B.
The random allocation of participants to Group A or Group B resulted in 42 participants allocated to Group A and 53 participants allocated to Group B.

The levels of the factors measured in the experiment were:

\begin{itemize}
\tightlist
\item
  Map type: \emph{Choropleth, Hexagon tile}
\item
  Trend: \emph{Locations in three population centres, Locations in multiple population centres, South-East to North-West}
\end{itemize}

Each group did not see the same data for both map types. We generated four simulated sets of data for each treatment.
This will generate twenty-four lineups (twelve lineups of geographic maps and twelve lineups of hexagon tile maps). Participants evaluated twelve lineups, size of each map type.
For each of the six geographic displays and six hexagon displays, two of each trend model were shown to the participants. We outline the design in Table. \ref{tab:exp-design}.

\begin{table}

\caption{\label{tab:exp-design}The experimental design results in the following allocation of replicate data sets within each trend model to groups A and B.}
\centering
\begin{tabular}[t]{llll}
\toprule
Group & Trend & Choro. & Hex.\\
\midrule
 & NW-SE & 1, 2 & 3, 4\\
\cmidrule{2-4}
 & Three Cities & 1, 2 & 3, 4\\
\cmidrule{2-4}
\multirow{-3}{*}{\raggedright\arraybackslash A} & All Cities & 2, 4 & 1, 3\\
\cmidrule{1-4}
 & NW-SE & 3, 4 & 1, 2\\
\cmidrule{2-4}
 & Three Cities & 3, 4 & 1, 2\\
\cmidrule{2-4}
\multirow{-3}{*}{\raggedright\arraybackslash B} & All Cities & 1, 3 & 2, 4\\
\bottomrule
\end{tabular}
\end{table}

The variables measured as a result of the changes were the probability of detection each display and the time taken to submit responses.
To measure the accuracy of the detections, the plot chosen for each lineup evaluated was compared to the position of the real spatial trend plot in the lineup. A correct result occurs when the chosen plot matches the position of the real plot, this was recorded in an additional binary variable; 1 = correct; 0 = incorrect.
High efficiency occurs when a small amount of time is taken to evaluate each lineup. This will be measured as the numeric variable measuring the length of time taken to submit the answers to the evaluation of each line up.

\hypertarget{simulation-process}{%
\subsection{Simulation process}\label{simulation-process}}

The lineup protocol allows a known model to be imposed on the null data set plots.
The underlying spatial correlation model was created to provide spatial autocorrelation between neighboring areas using the longitude and latitude values for the Statistical Areas.

\[z = 1\]
\[locations = longitude + latitude\]

The null model imposed suggests that neighbors are related. The randomness imposed was smoothed to mirror the practice taken by the Australian Cancer Atlas. This smoothing allowed neighbors to be related to each other and show distributions similar to the Liver cancer distribution shown in \ref{fig:liver}.

Spatially dependent data sets were simulated using the variogram model on the centroids of each geographic unit. Twelve simulations from the variogram model were created for each of the twelve lineups.

In these 12 sets of data, each of the 144 maps were smoothed several times to replicate the spatial autocorrelation seen in cancer data sets presented in the Australian Cancer Atlas.

For each of the 144 individual maps, the values attributed to each geographic area are rescaled to show a similar color scale from deep blue to dark red within each map.

A random location was selected for each set of lineup data.
In this location, a trend model was overlaid on the null set of spatially correlated data.
Each set of lineup data was used to produce a choropleth maps and hexagon tile maps. These matched pairs were split between Group A and Group B.

\hypertarget{experiment-procudure-and-data-collection}{%
\subsection{Experiment procudure and data collection}\label{experiment-procudure-and-data-collection}}

The participant answered demographic questions and provided consent before evaluating the lineups.

Demographics were collected regarding the study participants:

\begin{itemize}
\tightlist
\item
  Gender (female / male / other),
\item
  Degree education level achieved (high school / bachelors / masters / doctorate / other),
\item
  Age range (18-24 / 25-34 / 35-44 / 45-54 / 55+ / other)
\item
  Lived at least for one year in Australia (Yes / No )
\end{itemize}

Participants then moved to the evaulation phase.
The set of images differed for Group A and Group B.
After being allocated to a group, each individual was shown the 12 displays in randomised order.

Three questions were asked regarding each display:

\begin{itemize}
\tightlist
\item
  Plot choice
\item
  Reason
\item
  Difficulty
\end{itemize}

After completing the 12 evaluations, the participants were asked to submit their responses.

Data was collected through a web application containing the online survey.
Each participant used the internet to access the survey.
The data collection took place using a secure link between the survey web application and the googlesheet used to store results. The application would first connect to the googlesheet using the googlesheets {[}\protect\hyperlink{ref-sheets}{12}{]} R package, and interacted again at the completion of the survey by adding the participant's responses to the 12 displays as 12 rows of data in the googlesheet.

\hypertarget{experimental-design}{%
\subsection{Experimental design}\label{experimental-design}}

The choropleth map was used as the comparative visualization for presenting the lineups {[}\protect\hyperlink{ref-VVSIALM}{13}{]} as this is the common display for spatial cancer data.
Geographic distributions usually have some degree of spatial autcorrelation between neighbors.
This feature was incorporated in all maps shown in the lineup displays, the map that contained the trend feature shown in only one set of data was also affected by spatial autocorrelation.
A line up protocol was implemented to arrange \(N\) maps in each display.
A reasonable amount of null plots \(N-1\) in the lineup was chosen to ensure the real data map was well hidden. A reasonable number of plots to show in each lineup, \(N = 12\) was chosen to not overwhelm participants due to the detailed choropleth maps of Australian SA3 areas.

The hypotheses for each lineup are
\(H_0\) : All plots look the same
\(H_a\) : One plot looks different to the other plots

The same data was visualized on a choropleth map, and on a hexagon tile map, however in this study the participants only saw one of these two displays. The accuracy and times taken were aggregated for Groups and.
Comparing the results of participants who see the choropleth to those who see a hexagon tile map will show that population related distributions are spotted more frequently in a hexagon tile map display.

Let \(n\) be the number of independent observers and \(x_i\) the
number of observers who picked plot \(i\), \(i = \{1,...,m\}\)

Then \(x_i, x_2, ..., x_m\) follows a multinomial
distribution\(Mult_{\pi_1, \pi_2, ...., \pi_m}(x_i, x_2, ..., x_m)\) with
\(\sum_i \pi_i = 1\), where \(\pi_i\) is the probability that plot \(i\) is picked by an
observer, which we can estimate as \(\hat{\pi}_i = x_i/n\).
The researchers compared the length of time taken, and the accuracy of the participants choices.
The power of a lineup can therefore be estimated as the ratio of correct
identifications \(x\) out of \(n\) viewings.

\hypertarget{the-methods-of-data-analysis-used}{%
\subsection{The methods of data analysis used}\label{the-methods-of-data-analysis-used}}

The data analysis methods used in order to analyse and collate the results included downloading the survey submissions and opening them into the analysis software R {[}\protect\hyperlink{ref-RCore}{14}{]}.

For each of the 12 lineup displays the researchers calculated:

\begin{itemize}
\tightlist
\item
  accuracy: the proportion of subjects who detected the data plot
\item
  efficiency: average time taken to respond
\end{itemize}

\hypertarget{visualizations}{%
\subsubsection{visualizations}\label{visualizations}}

Side-by-side dot plots were made of accuracy (efficiency) against map type, facetted by trend model type.

Similar plots were made of the feedback and demographic variables - reason for choice, reported difficulty, gender, age, education, having lived in Australia - against the design variables.

Plots will be made in R (R Core Team 2019), with the ggplot2 package (Wickham 2016).

\hypertarget{modeling}{%
\subsubsection{Modeling}\label{modeling}}

The likelihood of detecting the data plot in the lineup can be modelled using a linear mixed effects model.
The R {[}\protect\hyperlink{ref-RCore}{14}{]} \texttt{glmer()} function in the {[}\protect\hyperlink{ref-lme4}{15}{]} package implements generalised linear mixed effect models. The model used includes the two main effects map type and trend model, which gives the fixed effects model to be:

\[\widehat{y_{ij}} = \mu + \tau_i + \delta_j + (\tau\delta)_{ij} + \epsilon_{i,j}, ~~~ i=1,2; ~j=1,2,3\]

where \(y_{ij} = 0, 1\) whether the subject detected the data plot, \(\mu\) is the overall mean, \(\tau_i, i=1,2\) is the map type effect, \(\delta_j\) is the trend model effect. We are allowing for an interaction between map type and trend model. Because the response is binary, a logistic model is used. This model can account for each individual participants' abilities as it includes a subject-specific random intercept. As each participant provides results from 12 lineups.

The model specifies a logistic link, this means the predicted values from the \texttt{glmer} model should be back-transformed to fit between 0 and 1. They are transformed with the link specified below:

\[\mu = \frac{e^{\eta}}{1 + e^{\eta}}\] \label{eq:transform}
\[\eta = f(\tau_i,\delta_j)\]

The feedback and demographic variables will possibly be incorporated as covariates.

Computation will be done using R {[}\protect\hyperlink{ref-RCore}{14}{]}, with the \texttt{lme4} package {[}\protect\hyperlink{ref-lme4}{15}{]}.

\hypertarget{limitations-of-the-data-collection}{%
\subsection{Limitations of the data collection}\label{limitations-of-the-data-collection}}

A pilot study was conducted to determine whether the lineups were appropriate. All lineups in the pilot study were deemed viable as at least one participant detected the real data plot in each lineup.
The demographics of the participants showed a skew towards male participants.
The randomness of the group allocation also resulted in more participants being allocated to Group B. Due to the allocation of lineup displays the participants all saw six Choropleth displays and six Hexagon Tile Map displays.

\hypertarget{results}{%
\section{Results}\label{results}}

Responses from 95 participants were collected. Three participants provided no answers for any task, and their data was removed. Set A was evaluated by 42 participants, and 53 evaluated set B. This resulted in 1104 evaluations, corresponding to 92 subjects, each evaluating 12 lineups, that are analysed on accuracy and speed. The certainty and reasons of subjects in their answers is also examined.

\hypertarget{participant-demographics}{%
\subsection{Participant demographics}\label{participant-demographics}}

Of the 92 participants, 67 were male, and 25 female. Most participants (56) had a Bacherlors degree, 13 had a Masters degree, and the remaining 23 had high school diplomas.

\hypertarget{accuracy}{%
\subsection{Accuracy}\label{accuracy}}

Fig. \ref{fig:detect-compare} displays the average detection rates for the two types of plot separately for each trend model. Each trend model was tested using four repetitions, evaluations on the same data set were seen as either choropleths or hexagon tile maps by each group as specified in Table. \ref{tab:exp-design}; the detection rates for each display are connected by a line segment. The Three Cities and All Cities trend models shown in the hexagon tile map allowed viewers to detect the data plot substantially more often than the choropleth counterparts. One replicate for the All Cities group, had similar detection rates for both plot type. Surprisingly, participants could also detect the gradual spatial trend in the NW-SE group from the hexagon tile map. We expected that the choropleth map would be superior for the type of spatial pattern, but the data suggests the hexagon tile map performs slightly better, or equally as well.

\begin{figure}
\centering
\includegraphics{paper_files/figure-latex/detect-compare-1.pdf}
\caption{\label{fig:detect-compare}The detection rates achieved by participants are contrasted when viewing the four replicates of the three trend models. Each point shows the probability of detection for the lineup display, the facets separate the trend models hidden in the lineup. The points for the same data set shown in a choroleth or haxgon tile map display are linked to show the difference in the detection rate.}
\end{figure}

Table. \ref{tab:desc-stats} shows the means and standard deviations of the detection rate for each type of plot and each trend model. This also gives the standard errors, the smallest standard deviation for all sets of replicates is the Three Cities trend model shown in a Choropleth display. This group of displays had a very small detection rate of 0.04. The mean detection rate for the Three Cities trend model shown as choropleth map lineups was also the smallest at 0.39.
The North-West to South-East (NW-SE) trend model unexpectedly had a higher mean detection rate for the hexagon tile map displays, but the difference in the means of detection rate was only 0.10.

\begin{table}

\caption{\label{tab:desc-stats}The rate of detection for each trend model has been calculate for the choropleth and hexagon tile map displays. The associated standard errors are also included.}
\centering
\begin{tabular}[t]{lccc}
\toprule
Type & NW-SE & Three Cities & All Cities\\
\midrule
Choro. & 0.51 & 0.04 & 0.23\\
 & (0.50) & (0.19) & (0.42)\\
\addlinespace
Hex. & 0.61 & 0.39 & 0.57\\
 & (0.49) & (0.49) & (0.50)\\
\bottomrule
\end{tabular}
\end{table}

Table. \ref{tab:detect-glmer1} presents a summary of the generalised linear mixed effects model, testing the effect of plot type and trend model on the detection rate. The results support the summary from Fig. \ref{fig:detect-compare} and all terms are statistically significant despite the large standard deviations observed in Table. \ref{tab:desc-stats}. Overall, the hexagon tile map performs marginally better than the choropleth for all trend models, which is a pleasant surprise. Allowing for the interaction effect, the difference in detection rate decreases for population related displays for a choroleth map lineup, but increases for a hexagon tile map display.
The log odds of detection show in Table. \ref{tab:detect-glmer1} can be back transformed after taking the sum of all terms for the trend and type of display that are of interest.
For the NW-SE distribution, the predicted detection rate for the hexagon tile map display increases the predicted probability of detection to 0.62 from 0.51 for choropleths, this is almost exactly the difference seen in the table of means and is significant only at the 0.05 level.

When a choropleth map display is used, the predicted detection rate for the Three Cities trend, 0.03; this is extremely low, especially compared to the NW-SE trend of 0.51.
When the All Cities trend is presented in a choropleth display the predicted probability of detection is 0.21.
The hexagon tile map has a substantially high detection rate for the display of a Three Cities trend 0.38 and All Cities trend 0.58.

\begin{table}

\caption{\label{tab:detect-glmer1}The model output for the generalised linear mixed effect model for detection rate. This model considers the type of display, the trend model hidden in the data plot, and accounts for contributor performance.}
\centering
\begin{tabular}[t]{rrlrr}
\toprule
Term & Est. & Sig. & Std. Error & P val\\
\midrule
Intercept & 0.02 & $^{ }$ & 0.17 & 0.90\\
Hex. & 0.46 & $^{*}$ & 0.22 & 0.04\\
\addlinespace
Three Cities & -3.43 & $^{***}$ & 0.42 & 0.00\\
All Cities & -1.35 & $^{***}$ & 0.24 & 0.00\\
\addlinespace
Hex:Three Cities & 2.46 & $^{***}$ & 0.47 & 0.00\\
Hex:All Cities & 1.17 & $^{***}$ & 0.33 & 0.00\\
\bottomrule
\end{tabular}
\end{table}

\hypertarget{speed}{%
\subsection{Speed}\label{speed}}

Fig. \ref{fig:beeswarm} shows horizontally jittered dot plots to contrast the time taken by participants to evaluate each lineup when viewing each type of display. The time are also separated by trend model and whether the data plot was detected or not detected. The time taken to complete an evaluation ranged from milliseconds to 60 seconds. The average time taken for type of display is shown as a large colored dot on each plot. when considering the heights of the green and orange dots, there is little difference in the average time taken to read a choropleth or hexgon tile map. Comparing the same colored dot across each trend model row, there is a slight increase in the time taken to correctly detected the data plot in the hexagon tile map lineup, but little difference in evaluation time for the choropleth display. However, there were substantially less correct detections for choropleth lineups for the Three cities and All Cities trends.

\begin{figure}
\centering
\includegraphics{paper_files/figure-latex/beeswarm-1.pdf}
\caption{\label{fig:beeswarm}The distribution of the time taken (seconds) to submit a response for each combination of trend, whether the data plot was detected, and type of display, shown using horizontally jittered dotplots. The colored point indicates average time taken for each plot type. Although some participants take just a few seconds per evaluation, and some take as much as mcuh as 60 seconds, but there is very little difference in time taken between plot types.}
\end{figure}

\hypertarget{certainty}{%
\subsection{Certainty}\label{certainty}}

Participants provided their level of certainty regarding their choice using a five point scale.
Unlike the accuracy and speed of reponses that were derived during the data processing phase, this was a subjective
assessment by the participant prompted by the question: `How certain are you about your choice?'.
Figure. \ref{certainty} shows the amount of times participants provided each level of certainty. This was separated for each combination of trend models and display type, and colored depending on whether a participant correctly detected the data plot in the lineup.
Participants chose 4 or 5 often when viewing the population related trends in the choropelth display, even though they were often incorrect when viewing an All Cities trend and overwhelmingly incorrect for the Three Cities trend. This shows overconfidence in their detection ability when using a choropleth map display. Participants were less likely to be certain when their choice was incorrect and they were viewing a hexagon tile map.
For each trend model, participants were more likely to doubt their choice and choose 1 or 2 in the hexagon tile map displays, even though many had made the correct choice.

\begin{figure}
\centering
\includegraphics{paper_files/figure-latex/certainty-1.pdf}
\caption{\label{fig:certainty}The amount of times each level of certainty was chosen by participants when viewing hexagon tile map or choropleth displays. Participants were more likely to choose a high certainty when considering a Choropleth map. The mid value of 3 was the default certainty, it was chosen most for the Hexagon tile map displays.}
\end{figure}

\hypertarget{reason}{%
\subsection{Reason}\label{reason}}

Participants were asked why they had made their plot choice and were able to select from a set of suggested reasons.
``Color trend across the areas'' was the most common selection for NW-SE trend displays.

The reasons used when viewing choropleth displays varied more than the hexagon tile map reasons.
The hexagon tile map displays resulted in ``Clusters of color'' as the most common choice made by participants.

The choice ``None of these reasons'' was used as the default value to minimise noise from participants who did not select a response.

\begin{table}

\caption{\label{tab:reason}The amount of participants that selected each reason for their choice of plot when looking at each trend model shown in Choropleth and Hexagon Tile maps. The facets show whether or not the choice was correct.}
\centering
\begin{tabular}[t]{llll}
\toprule
Trend & Detected & Choro. & Hex.\\
\midrule
 & No & trend & clusters\\
\cmidrule{2-4}
\multirow{-2}{*}{\raggedright\arraybackslash NW-SE} & Yes & trend & clusters\\
\cmidrule{1-4}
 & No & trend & clusters\\
\cmidrule{2-4}
\multirow{-2}{*}{\raggedright\arraybackslash Three Cities} & Yes & consistent & clusters\\
\cmidrule{1-4}
 & No & trend & clusters\\
\cmidrule{2-4}
\multirow{-2}{*}{\raggedright\arraybackslash All Cities} & Yes & clusters, consistent & clusters\\
\bottomrule
\end{tabular}
\end{table}

\hypertarget{discussion}{%
\section{Discussion}\label{discussion}}

The intention of this study was to contrast the use of the choropleth map and the hexagon tile map.
The visual inference lineup protocol was employed to contrast the effectiveness of the displays.
The results have shown that overall the use of the hexagon tile map display allows participants to find the data plot in the lineup more often.
Using the visual inference protocol this result can be extended to show that it is a valid alternative display to communicate spatial distributions of population related data.

We expected that the choropleth map would be superior for communicating the spatial pattern of geographic distributions. The data suggest that the participants perform slightly better or equally as well for each replicate in each trend model across the two displays. Table II shows that the difference in the mean detection rate for the two trend models was 0.10.

The differences seen in the compare detect plot and Table. \ref{tab:desc-stats} are reflected in the model results. Surprisingly the difference scene for the geographic distribution was significant at the 0.05 level.
It also showed that the hexagon tile map display performs marginally better than the choropleth for all trend models. Unexpectedly the detection rate suffers when using a choropleth map to display population related distributions.

While the significance of the difference in detection was the key focus of this experiment, the secondary focus was the time taken by participants. it was expected that the disciplines may take longer to consider the hexagon tile map distribution but would be able to detect the data plot in the lineup.
The bimodal distributions seen in the Fig. \ref{fig:choices} display showed very little difference in the mean evaluation times. As the ranges of all of the distributions approached 60 seconds it cannot be said that the participants' took longer to evaluate the hexagon tile map displays.

The responses to the questions asked of participants included the reason for their choice and the certainty around their choice.
Fig. \ref{fig:reason} shows high levels of certainty of 4 and 5 were chosen by participants when looking at the population distributions in a choropleth map display show that they were over confident when attempting to find the real data plot in the choropleth map displays. Participants performed better on the NW-SE distribution shown in the choropleth display and were reasonably confident about their decisions.
The high levels of the mid range value of 3 could indicate that the participant did not want to provide a response, as this wasy the default value. Those who chose level 4 or 5 were equally likely to be correct for the three cities lineups, but more likely to be correct than incorrect for the other two trend models.

The color scaling applied in Three cities and All cities displays resulted in the rural areas of the real data plot appearing more blue or yellow than the other plots in the lineups.
Due to the consisent coloring of rural areas in a choropleth display, the choice ``All areas have similar colors'' was most common reason for a participants choice. The All Cities displays colored the inner-city areas of all capital cities more red, this was observable to participants and explains the equal choice of the city clusters or rural color consistency.
Choosing ``Clusters of colour'' was expected when participants viewed the Hexagon tile map display of the All Cities and Three Cities distributions. It was unexpected that it was also the most common reason for the NW-SE hexagon tile map displays.
Due to the spatial covariance introduced in the smoothing, groups of similarly colored hexagons were present in all of the hexagon tile map displays. All Cities and Three Cities distributions of real data trends had distinctly different patterns or red inner-city areas, while some of the plots in each lineup may have shared similar features.

\hypertarget{conclusion}{%
\section{Conclusion}\label{conclusion}}

The choropleth map display and the tessellated hexagon tile map have been contrasted using the lineup protocol. The hexagon tile map was significantly more effective for spotting a real population related data trend model hidden in a lineup.

The hexagon tile map display should be considered as an alternative visualization method when communicating distributions that relate to the population across a set of geographic units. As an additional display to the familiar choropleth map, Cancer Atlas products may benefit from the opportunity to allow exploration via an alternative display. The spatial distributions used to test these displays were inspired by the real spatially smoothed estimates of the cancer burden on Australian communities. However, this technique may be extended to other population related distributions, such as other diseases.

The increasing population densities of capital cities despite large land area exaserbates the difference in the smallest and largest communities.
The population density structure of Australia can be considered similar to that of Canada, New Zealand and many others. Therefore, this display is not only relevant to Australia, but all nations or population distributions that experience densely populated cities separated by vast rural expanses.

\hypertarget{acknowledgment}{%
\section{Acknowledgment}\label{acknowledgment}}

The authors would like to thank the Australian Cancer Atlas team for discussions regarding alternative spatial visualizations. They would also like to thank Kerrie Mengersen and Dr.~Earl Duncan for suggestions and comments.

Ethics approval for the online survey was granted by QUT's Ethics Committee (Ethics Application Number: 1900000991). All applicants provided informed consent in line with QUT regulations prior to participating in this research.

\hypertarget{references}{%
\section{References}\label{references}}

\newpage

\hypertarget{acknowledgment-1}{%
\section{Acknowledgment}\label{acknowledgment-1}}

The authors would like to thank the Australian Cancer Atlas team for discussions regarding alternative spatial visualizations. They would also like to thank Distinguished Professor Kerrie Mengersen and Doctor Earl Duncan for suggestions and comments.

The source code to produce this document can be found on \href{https://github.com/srkobakian/experiment/tree/master/paper}{GitHub}.
Supplementary materials have been included to discuss the survey procedures,

The analysis of the work was completed in R {[}\protect\hyperlink{ref-RCore}{14}{]} with the use of the following packages:

\begin{itemize}
\tightlist
\item
  For document creation: rmarkdown {[}\protect\hyperlink{ref-rmarkdown}{16}{]}, rticles {[}\protect\hyperlink{ref-rticles}{17}{]}, knitr {[}\protect\hyperlink{ref-knitr}{18}{]}.
\item
  For lineup creation and data analysis: tidyverse {[}\protect\hyperlink{ref-tidyverse}{19}{]}, nullabor {[}\protect\hyperlink{ref-nullabor}{20}{]}, ggthemes {[}\protect\hyperlink{ref-ggthemes}{21}{]}, RColorBrewer {[}\protect\hyperlink{ref-RColorBrewer}{22}{]}.
\item
  For image displays: cowplot {[}\protect\hyperlink{ref-cowplot}{23}{]}, png {[}\protect\hyperlink{ref-png}{24}{]}, grid {[}\protect\hyperlink{ref-grid}{25}{]}.
\item
  For modeling and presentation of models: lme4 {[}\protect\hyperlink{ref-lme4}{15}{]}, kableExtra {[}\protect\hyperlink{ref-kableExtra}{26}{]}.
\end{itemize}

Ethics approval for the online survey was granted by QUT's Ethics Committee (Ethics Application Number: 1900000991). All applicants provided informed consent in line with QUT regulations prior to participating in this research.

\newpage

\hypertarget{references-1}{%
\section{References}\label{references-1}}

\hypertarget{refs}{}
\leavevmode\hypertarget{ref-sugarbag}{}%
1. Kobakian S, Cook D (2019) Sugarbag: Create tessellated hexagon maps.

\leavevmode\hypertarget{ref-atlas}{}%
2. Queensland CC Australian Cancer Atlas (https://atlas.cancer.org.au).

\leavevmode\hypertarget{ref-GIIV}{}%
3. Wickham H, Cook D, Hofmann H, Buja A (2010) Graphical inference for infovis. IEEE Transactions on Visualization and Computer Graphics (Proc InfoVis '10) 16:973--979

\leavevmode\hypertarget{ref-CBATCC}{}%
4. Kocmoud C, House D (1998) A Constraint-based Approach to Constructing Continuous Cartograms. In: Proc. Symp. Spatial data handling. pp 236--246

\leavevmode\hypertarget{ref-CTTMB}{}%
5. Griffin TL (1980) Cartographic Transformation of the Thematic Map Base. Cartography 11:163--174

\leavevmode\hypertarget{ref-NISCC}{}%
6. Dent BD (1972) A Note on the Importance of Shape in Cartogram Communication. Journal of Geography 71:393--401

\leavevmode\hypertarget{ref-ACTUC}{}%
7. Dorling D (2011) Area Cartograms: Their Use and Creation. In: Concepts and techniques in modern geography (catmog). pp 252--260

\leavevmode\hypertarget{ref-NAC}{}%
8. Olson JM (1976) Noncontiguous Area Cartograms. The Professional Geographer 28:371--380

\leavevmode\hypertarget{ref-RSCW}{}%
9. Raisz E (1963) Rectangular Statistical Cartograms of the World. Journal of Geography 35:8--10

\leavevmode\hypertarget{ref-CDWCS}{}%
10. Monmonier M (2005) Cartography: Distortions, World-views and Creative Solutions. Progress in Human Geography 29:217--224

\leavevmode\hypertarget{ref-GTPCCD}{}%
11. Hofmann H, Follett L, Majumder M, Cook D (2012) Graphical tests for power comparison of competing designs. IEEE Transactions on Visualization and Computer Graphics 18:2441--2448

\leavevmode\hypertarget{ref-sheets}{}%
12. Bryan J, Zhao J (2018) Googlesheets: Manage google spreadsheets from r.

\leavevmode\hypertarget{ref-VVSIALM}{}%
13. Majumder M, Hofmann H, Cook D (2013) Validation of visual statistical inference, applied to linear models. Journal of the American Statistical Association 108:942--956

\leavevmode\hypertarget{ref-RCore}{}%
14. R Core Team (2019) R: A language and environment for statistical computing. R Foundation for Statistical Computing, Vienna, Austria

\leavevmode\hypertarget{ref-lme4}{}%
15. Bates D, Mächler M, Bolker B, Walker S (2015) Fitting linear mixed-effects models using lme4. Journal of Statistical Software 67:1--48

\leavevmode\hypertarget{ref-rmarkdown}{}%
16. Xie Y, Allaire JJ, Grolemund G (2018) R markdown: The definitive guide. Chapman; Hall/CRC, Boca Raton, Florida

\leavevmode\hypertarget{ref-rticles}{}%
17. Allaire J, Xie Y, R Foundation, et al (2019) Rticles: Article formats for r markdown.

\leavevmode\hypertarget{ref-knitr}{}%
18. Xie Y (2014) Knitr: A comprehensive tool for reproducible research in R. Implementing reproducible computational research

\leavevmode\hypertarget{ref-tidyverse}{}%
19. Wickham H, Averick M, Bryan J, et al (2019) Welcome to the tidyverse. Journal of Open Source Software 4:1686

\leavevmode\hypertarget{ref-nullabor}{}%
20. Wickham H, Chowdhury NR, Cook D, Hofmann H (2018) Nullabor: Tools for graphical inference.

\leavevmode\hypertarget{ref-ggthemes}{}%
21. Arnold JB (2019) Ggthemes: Extra themes, scales and geoms for 'ggplot2'.

\leavevmode\hypertarget{ref-RColorBrewer}{}%
22. Neuwirth E (2014) RColorBrewer: ColorBrewer palettes.

\leavevmode\hypertarget{ref-cowplot}{}%
23. Wilke CO (2019) Cowplot: Streamlined plot theme and plot annotations for 'ggplot2'.

\leavevmode\hypertarget{ref-png}{}%
24. Urbanek S (2013) Png: Read and write png images.

\leavevmode\hypertarget{ref-grid}{}%
25. R Core Team (2019) R: A language and environment for statistical computing. R Foundation for Statistical Computing, Vienna, Austria

\leavevmode\hypertarget{ref-kableExtra}{}%
26. Zhu H (2019) KableExtra: Construct complex table with 'kable' and pipe syntax.

\end{document}


