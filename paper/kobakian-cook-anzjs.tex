% Options for packages loaded elsewhere
% Options for packages loaded elsewhere
\PassOptionsToPackage{unicode}{hyperref}
\PassOptionsToPackage{hyphens}{url}
\PassOptionsToPackage{dvipsnames,svgnames,x11names}{xcolor}
%
\documentclass[
doublespace,
  times]{anzsauth}
\usepackage{xcolor}
\usepackage{amsmath,amssymb}
\setcounter{secnumdepth}{5}
\usepackage{iftex}
\ifPDFTeX
  \usepackage[T1]{fontenc}
  \usepackage[utf8]{inputenc}
  \usepackage{textcomp} % provide euro and other symbols
\else % if luatex or xetex
  \usepackage{unicode-math} % this also loads fontspec
  \defaultfontfeatures{Scale=MatchLowercase}
  \defaultfontfeatures[\rmfamily]{Ligatures=TeX,Scale=1}
\fi
\usepackage{lmodern}
\ifPDFTeX\else
  % xetex/luatex font selection
\fi
% Use upquote if available, for straight quotes in verbatim environments
\IfFileExists{upquote.sty}{\usepackage{upquote}}{}
\IfFileExists{microtype.sty}{% use microtype if available
  \usepackage[]{microtype}
  \UseMicrotypeSet[protrusion]{basicmath} % disable protrusion for tt fonts
}{}
\makeatletter
\@ifundefined{KOMAClassName}{% if non-KOMA class
  \IfFileExists{parskip.sty}{%
    \usepackage{parskip}
  }{% else
    \setlength{\parindent}{0pt}
    \setlength{\parskip}{6pt plus 2pt minus 1pt}}
}{% if KOMA class
  \KOMAoptions{parskip=half}}
\makeatother
% Make \paragraph and \subparagraph free-standing
\makeatletter
\ifx\paragraph\undefined\else
  \let\oldparagraph\paragraph
  \renewcommand{\paragraph}{
    \@ifstar
      \xxxParagraphStar
      \xxxParagraphNoStar
  }
  \newcommand{\xxxParagraphStar}[1]{\oldparagraph*{#1}\mbox{}}
  \newcommand{\xxxParagraphNoStar}[1]{\oldparagraph{#1}\mbox{}}
\fi
\ifx\subparagraph\undefined\else
  \let\oldsubparagraph\subparagraph
  \renewcommand{\subparagraph}{
    \@ifstar
      \xxxSubParagraphStar
      \xxxSubParagraphNoStar
  }
  \newcommand{\xxxSubParagraphStar}[1]{\oldsubparagraph*{#1}\mbox{}}
  \newcommand{\xxxSubParagraphNoStar}[1]{\oldsubparagraph{#1}\mbox{}}
\fi
\makeatother


\usepackage{longtable,booktabs,array}
\usepackage{calc} % for calculating minipage widths
% Correct order of tables after \paragraph or \subparagraph
\usepackage{etoolbox}
\makeatletter
\patchcmd\longtable{\par}{\if@noskipsec\mbox{}\fi\par}{}{}
\makeatother
% Allow footnotes in longtable head/foot
\IfFileExists{footnotehyper.sty}{\usepackage{footnotehyper}}{\usepackage{footnote}}
\makesavenoteenv{longtable}
\usepackage{graphicx}
\makeatletter
\newsavebox\pandoc@box
\newcommand*\pandocbounded[1]{% scales image to fit in text height/width
  \sbox\pandoc@box{#1}%
  \Gscale@div\@tempa{\textheight}{\dimexpr\ht\pandoc@box+\dp\pandoc@box\relax}%
  \Gscale@div\@tempb{\linewidth}{\wd\pandoc@box}%
  \ifdim\@tempb\p@<\@tempa\p@\let\@tempa\@tempb\fi% select the smaller of both
  \ifdim\@tempa\p@<\p@\scalebox{\@tempa}{\usebox\pandoc@box}%
  \else\usebox{\pandoc@box}%
  \fi%
}
% Set default figure placement to htbp
\def\fps@figure{htbp}
\makeatother





\setlength{\emergencystretch}{3em} % prevent overfull lines

\providecommand{\tightlist}{%
  \setlength{\itemsep}{0pt}\setlength{\parskip}{0pt}}



 
\usepackage[]{natbib}
\bibliographystyle{anzsj}


\usepackage{pmboxdraw}
\usepackage{booktabs}
\usepackage{longtable}
\usepackage{array}
\usepackage{multirow}
\usepackage{wrapfig}
\usepackage{float}
\usepackage{colortbl}
\usepackage{pdflscape}
\usepackage{tabu}
\usepackage{threeparttable}
\usepackage{threeparttablex}
\usepackage[normalem]{ulem}
\usepackage{makecell}
\usepackage{xcolor}
\makeatletter
\@ifpackageloaded{caption}{}{\usepackage{caption}}
\AtBeginDocument{%
\ifdefined\contentsname
  \renewcommand*\contentsname{Table of contents}
\else
  \newcommand\contentsname{Table of contents}
\fi
\ifdefined\listfigurename
  \renewcommand*\listfigurename{List of Figures}
\else
  \newcommand\listfigurename{List of Figures}
\fi
\ifdefined\listtablename
  \renewcommand*\listtablename{List of Tables}
\else
  \newcommand\listtablename{List of Tables}
\fi
\ifdefined\figurename
  \renewcommand*\figurename{Figure}
\else
  \newcommand\figurename{Figure}
\fi
\ifdefined\tablename
  \renewcommand*\tablename{Table}
\else
  \newcommand\tablename{Table}
\fi
}
\@ifpackageloaded{float}{}{\usepackage{float}}
\floatstyle{ruled}
\@ifundefined{c@chapter}{\newfloat{codelisting}{h}{lop}}{\newfloat{codelisting}{h}{lop}[chapter]}
\floatname{codelisting}{Listing}
\newcommand*\listoflistings{\listof{codelisting}{List of Listings}}
\makeatother
\makeatletter
\makeatother
\makeatletter
\@ifpackageloaded{caption}{}{\usepackage{caption}}
\@ifpackageloaded{subcaption}{}{\usepackage{subcaption}}
\makeatother
\usepackage{bookmark}
\IfFileExists{xurl.sty}{\usepackage{xurl}}{} % add URL line breaks if available
\urlstyle{same}
\hypersetup{
  pdftitle={Comparing the Effectiveness of the Choropleth Map with a Hexagon Tile Map for Communicating Cancer Statistics},
  pdfauthor={Stephanie Kobakian; Dianne Cook},
  pdfkeywords={statistics, visual inference, geospatial, population},
  colorlinks=true,
  linkcolor={blue},
  filecolor={Maroon},
  citecolor={Blue},
  urlcolor={Blue},
  pdfcreator={LaTeX via pandoc}}

\usepackage{moreverb}
\usepackage{url}
\usepackage{grffile}
\usepackage[UKenglish]{isodate}

%%%%%%%%%%%%%%%%%%%%%%%%%%%%%%%%%%%%%%%%%%%%%%%%%%%%%%%%%%%%%%%%%%%%%%%%%%%%%%

% The year in the following may need to be updated!
\def\volumeyear{2025}

% The following command ("obviously") effects line numbering
% of the document.

\usepackage{lineno}
\linenumbers


\def\firstletters{\bgroup \catcode`-=10 \catcode`(=10 \filA}
\def\filA#1{\filB#1 {\end} }
\def\filB#1#2 {\ifx\end#1\egroup \else#1 \expandafter\filB\fi}

\runningheads{
Automated Residual Plot Assessment
}{
\firstletters{STEPHANIE}KOBAKIAN, AND  \firstletters{DIANNE}COOK
}

\title{Comparing the Effectiveness of the Choropleth Map with a Hexagon
Tile Map for Communicating Cancer Statistics}

\author{
Stephanie Kobakian\addressnum{1} and
Dianne Cook\addressnum{2}
}
\affiliation{
Queensland University of Technology and
Monash University
}

\address{
\addressnum{1} Science and Engineering Faculty, Queensland University of
Technology, 2 George St, Brisbane, Australia\\
\addressnum{2} Econometrics and Business Statistics Faculty, Monash
University, 29 Ancora Imparo Way, Clayton, VIC 3800, Australia\\
\hspace*{1ex} Email:  
}


\date{2025}
\begin{document}

\begin{abstract}
The choropleth map display is commonly used for communicating spatial
distributions across geographic areas. However, when choropleths are
used the size of the geographic units will influence the understanding
of the distribution derived by map users. The hexagon tile map is
presented as an alternative display for visualizing population related
distributions effectively. Visual inference is used to measure the power
of the hexagon tile map design, and the choropleth is used as a
comparison. The hexagon tile map display is tested using a distribution
that is directly related to the geography, with values monotonically
increasing from the North-West to South-East areas of Australia. This
study finds in a hexagon tile map lineup the single map that contains a
population related distribution is detected with greater probability
than the same data displayed in a choropleth map. These findings should
encourage map creators to implement alternative displays and consider a
hexagon tile map when presenting spatial distributions of heterogeneous
areas.
\end{abstract}

\keywords{statistics; visual inference; geospatial; population}
          

\maketitle


\section{Introduction}\label{sec-introduction}

This study compares the effectiveness of the spatial display, a hexagon
tile map, against the standard, a choropleth map, for communicating
information about disease statistics. The choropleth map is the
traditional method for visualizing aggregated statistics across
administrative boundaries. The hexagon tile map builds on existing
displays, such as the cartogram, and tessellated hexagon displays. A
hexagon tile map forgoes the familiar boundaries, in favor of
representing each geographic unit as an equally sized hexagon, placed
approximately in the correct spatial location. It differs in the relaxed
requirement to have connected hexagons, and allows sparsely located
hexagons. This type of display may be useful for other countries, and
other purposes. The algorithm to construct a hexagon tile map is
available in the R package sugarbag \citep{sugarbag}.

The hexagon tile map was designed for Australia, motivated by a need to
display spatial statistics for the Australian Cancer Atlas. None of the
existing approaches for creating cartograms or hexagon tiling perform
well for the Australian landscape, which has vast open spaces and
concentrations of population in small regions clustered on the
coastlines.

The Australian Cancer Atlas \citep{atlas} is an online interactive web
tool created to explore the burden of cancer on Australian communities.
There are many cancer types to be explored individually or aggregated.
The Australian Cancer Atlas allows users to explore the patterns in the
distributions of cancer statistics over the geographic space of
Australia. It uses a choropleth map display and diverging color scheme
to draw attention to relationships between neighboring areas. The
hexagon tile map may be a useful alternative display to enhance the
atlas.

The experiment was conducted using the lineup protocol, a visual
inference procedure \citep{GIIV}, to objectively test the effectiveness
of the two displays.

The paper is organised as follows. The next section discusses the
background of geographic data display and visual inference procedures.
The Section~\ref{sec-methodology} describes the methods for conducting
the experiment and analysing the results. The results are summarized in
the Section~\ref{sec-results}, followed by a discussion about the
broader implications for the use of this map style.

\section{Background}\label{sec-background}

\subsection{Spatial data displays}\label{spatial-data-displays}

Spatial visualisations communicate the distribution of statistics over
geographic landscapes. The choropleth map \citep[\citet{BCM}]{EI} is a
traditional display. It is used to present statistics that have been
aggregated on geographic units. Creating a choropleth map involves
drawing polygons representing the administrative boundaries, and filling
with colour mapped to the value of the statistic. The choropleth map
places the statistic in the context of the spatial domain, so that the
reader can see whether there are spatial trends, clusters or anomalies.
This is important for digesting disease patterns. If there is a trend it
may imply that the disease is spreading from one location to another. If
there is a cluster, or an anomaly, there may be a localized outbreak of
the disease. Aggregating the statistic on administrative units, provides
a level of privacy to individuals, while allowing the impact of the
disease on the community to be analyzed.

\begin{figure}

\centering{

\includegraphics[width=1\linewidth,height=\textheight,keepaspectratio]{kobakian-cook-anzjs_files/figure-pdf/fig-thyroid-1.pdf}

}

\caption{\label{fig-thyroid}Thyroid incidence among females across the
Statistical Areas of Australia at Level 2, displayed using a choropleth
(a) and a hexagon tile map (b). Blue indicates lower than average, and
red indicates higher than average incidence. The choropleth suggests
high incidence is clustered on the east coast but misses the high
incidence in Perth and a few locations in inner Melbourne visible in the
hexagon tile map.}

\end{figure}%

The choropleth map is an effective spatial display if the size of the
geographic units is relatively uniform. This is not the case for most
countries. Size heterogeneity in administrative units is particularly
extreme in Australia: most of the landscape of Australia is sparsely
settled, with the population densely clustered into the narrow coastal
strips. Figure~\ref{fig-thyroid} a shows the choropleth map of thyroid
cancer rates in Australia. The choropleth map focuses attention on the
geography, and for heterogeneously sized areas it presents a biased view
of the population related distribution of the statistic \citep{CBATCC}.
\emph{Land does not get cancer, people do} -- a more effective way to
communicate the spatial distributions of cancer statistics is needed
(sentiment motivated by \citet{monmonier2018how}).

A cartogram is a general solution for adequately displaying a
population-based statistic. It transforms the geographic map base to
reflect the population in the geographic region, while preserving some
aspects of the geographic location. There are several cartogram
algorithms \citep[\citet{CBATCC}]{ACTUC}; each involves shifting the
boundaries of geographic units, using the value of the statistic to
increase or decrease the area taken by the geographic unit on the map.
The changes to the boundaries result in cartograms that accurately
communicate population by map area for each of the geographic units but
can result in losing the familiar geographic information. For Australia,
the transformations warp the country so that it is no longer
recognizable.

Alternative algorithms make various trade offs between familiar shapes
and representation of geographic units. The non-contiguous cartogram
method \citep{NAC} keeps the shapes of geographic units intact, and
changes the size of the shape. This method disconnects areas creating
empty space on the display losing the continuity of the spatial display
of the statistic. The Dorling cartogram \citep{ACTUC} represents each
unit as a circle, sized according to the value of the statistic. The
neighbour relationships are mostly maintained by how the circles touch.
A similar approach was pioneered by \citet{RSCW}, using rectangles that
tile to align borders of neighbours \citep{CDWCS}. There have been
thorough reviews of the array of methods, as suitable for cancer atlas
displays \citep[\citet{BCM}]{review}.

The hexagon tile map algorithm, automatically matches spatial regions to
their nearest hexagon tile, from a grid of tiles. It has the effect of
spreading out the inner city areas while maintaining the spatial
locations or regions in remote areas. The algorithm is available in the
R package, sugarbag \citep{sugarbag}. Figure~\ref{fig-thyroid} b shows
the hexagon tile map, where the map is coloured from substantially below
average (blue) to substantially above average (red) rates. The inner
city areas have expanded, making it possible to see the cancer incidence
in the small, densely populated areas. Remote regions are represented by
isolated hexagons, which is not ideal, but maintains the spatial
location of these data values. It is of interest to know how well the
spatial distribution is perceived for this display, in comparison to the
choropleth.

\subsection{Visual Inference}\label{visual-inference}

In order to assess the effectiveness of the hexagon tile map, the lineup
protocol \citep[\citet{GIIV}]{BCHLLSW09} from visual inference
procedures is employed. The approach mirrors classical statistical
inference. The procedures for doing a power comparison of competing plot
designed, outlined in \citet{GTPCCD}, are followed.

In classical statistical inference hypothesis testing is conducted by
comparing the value of a test statistic on a standard reference
distribution, computed assuming the null hypothesis is true. If the
value is extreme, the null hypothesis is rejected, because the test
statistic value is unlikely to have been so extreme if it was true. In
the lineup protocol, the plot plays the role of the test statistic, and
the data plot is embedded in a field of null plots. Defining the plot
using a grammar of graphics \citep{ggplot2} makes it a functional
mapping of the variables and thus, it can be considered to be a
statistic. With the same data, two different plots can be considered to
be competing statistics, one possibly a more powerful statistic than the
other.

Hypothesis testing with the lineup protocol requires human evaluation.
The human judge is required to identify the most different plot among
the field of plots. If this corresponds to the data plot -- the test
statistic -- the null hypothesis is rejected. It means that the data
plot is extreme relative to the reference distribution of null plots.

The null hypothesis is explicitly provided by the grammatical plot
description. For example, if a histogram is the map type being used, the
null might be that the underlying distribution of the data is a
Gaussian. Null data would be generated by simulating from a normal
model, with the same mean and standard deviation as the data. In
practice, the null hypothesis used is generic, such as \emph{there is NO
structure or a pattern in the plot}, and contrasted to an alternative
that there is structure.

The chance that an observer picks the data plot out of a lineup of size
\(m\) plots accidentally, if the null hypothesis is true is \(1/m\).
With \(K\) observers, the probability of \(k\) randomly choosing the
data plot, roughly follows a binomial distribution with \(p=1/m\).
Figure~\ref{fig-lineup} shows a lineup of the hexagon tile map, of size
\(m=12\). Plot 3 is the data plot, and the remaining 11 are plots of
null data.

\begin{figure}

\centering{

\includegraphics[width=1\linewidth,height=\textheight,keepaspectratio]{kobakian-cook-anzjs_files/figure-pdf/fig-lineup-1.pdf}

}

\caption{\label{fig-lineup}This lineup of twelve hexagon tile map
displays contains one map with a real population related structure. The
rest are null plots that contain only spatial dependence.}

\end{figure}%

In order to determine the effectiveness of a type of display, this
probability is less relevant than the overall proportion of observers
who pick the data plot, \(k/K\). The power of the test statistic (data
plot) is provided by this proportion. Power in a statistical sense is
the ability of the statistic to \emph{produce a rejection} of the null
hypothesis, if it is indeed \emph{not true}. With the same data plotted
using two different displays, the display with the highest proportion of
people who choose the data plot would be considered to be the most
powerful statistic.

\section{Methodology}\label{sec-methodology}

This study aims to answer two key questions around the presentation of
spatial distributions:

\begin{enumerate}
\def\labelenumi{\arabic{enumi}.}
\tightlist
\item
  Are spatial disease trends that impact highly populated small areas
  detected with higher accuracy, when viewed in a hexagon tile map?
\item
  Are people faster in detecting spatial disease trends that impact
  highly populated small areas when using a hexagon tile map?
\end{enumerate}

Additional considerations when completing this experimental task
included the difficulty experienced by participants and the certainty
they had in their decision.

Australia is used for the study, with Statistical Area 3 (SA3)
\citep{abs2016} as the geographic units. The results should apply
broadly to any other geographic area of interest.

\subsection{Experimental factors}\label{experimental-factors}

The primary factor in the experiment is the map type. The secondary
factor is a trend model. Three trend models were developed, one
mirroring a large spatial trend for which the choropleth would be
expected to do well, and two with differing level of inner city hot
spots. These latter two reflect the structure seen in the thyroid cancer
data (Figure~\ref{fig-thyroid}). This produces six treatment levels:

\begin{itemize}
\tightlist
\item
  Map type: \emph{Choropleth, Hexagon tile}
\item
  Trend: \emph{South-East to North-West; Locations in three population
  centres; Locations in multiple population centres, }
\end{itemize}

Data is generated for each of the trend models, with four replicates,
and each displayed both as a choropleth and as a hexagon tile map, which
yields 12 data sets, and 24 data plots. This set of displays is divided
in half, providing two sets of 12 displays, Group A and Group B.
Participants were randomly allocated to Group A or B. Participants saw a
data set only once, either as a choropleth or as a hexagon tile map.
Figure~\ref{fig-exp-design} summarises the design and the allocation of
the displays.

\begin{figure}

\centering{

\includegraphics[width=0.8\linewidth,height=\textheight,keepaspectratio]{kobakian-cook-anzjs_files/figure-pdf/fig-exp-design-1.pdf}

}

\caption{\label{fig-exp-design}The experimental design used in the
study. Participants are allocated to with group A or B, to evaluate
either the choropleth or hexagon tile map lineup of each simulated data
set.}

\end{figure}%

\subsection{Generating null data}\label{generating-null-data}

Null data needs to be data with no (interesting) structure. In most
scenarios, permutation is the main approach for generating null plots.
It is used to break association between variables, while maintaining
marginal distributions. This is too simple for spatial data. In spatial
data, a key feature is the spatial dependence or smoothness over the
landscape. To do something simple, like permute the values relative to
the geographic location would produce null plots which are too chaotic,
and the data plot will be recognisable for its smoothness rather than
any structure of interest.

For spatial data, null data is stationary data, where the mean, variance
and spatial dependence are constant over the geographic units.
Stationary data is specified by a variogram model \citep{POG}.
Simulating from a variogram model, where the spatial dependence is
specified, generates the stationary spatial data used for the null
plots. The parameters for the Gaussian model were sill=1, range=0.3 with
the variance generated by a standard normal distribution.

The R package \texttt{gstat} \citep{gstat} was used to simulate 144 null
sets, 12 data sets for each plot in a lineup, and 12 sets for 12
lineups. Each null set was further processed to produce maps with
similar spatial dependence as those produced by the methods in
\citet{atlas-methods}, by averaging a small number of spatial
neighbours.

\subsection{Generating lineups}\label{generating-lineups}

For each trend model, four real data displays were created by
manipulating the centroid values of each of the SA3 geographic units.
Each trend model is motivated by patterns observed in spatial data:
North West to South East (NW-SE) is a basic spatial trend across the
entire country, Three Cities is the existence of clusters of high
values, and All Cities is also clusters, but more of them. We would
expect that clusters pattern to be more visible with a hexagon tile map
but the large spatial trend to be more visible in the choropleth map.

The NW-SE distribution was created using a linear equation of the
centroid longitude and latitude values. The All Cities trend model was
created using the distance from the centroid of each geographic unit to
the closest capital city in Australia, calculated when creating the
hexagon tile map produced by the sugarbag \citep{sugarbag} package.
Two-thirds of SA3s (201/336) were considered greater capital city areas,
the values of these areas were increased to create red clusters. The
amount was chosen to make clusters around the cities visible, even in
the choropleth with careful inspection. A similar selection process was
applied to the Three Cities' trend model. However, for each of the four
replicates for the Three Cities trend, a random sample of capital cities
was taken from Sydney, Brisbane, Melbourne, Adelaide, Perth, and Hobart.
Only values of the areas nearest to the three cities were increased to
create clusters.

One of the lineup locations was chosen to embed the real trend model
map, in each of the four replicates, for the three trend models. The
locations were chosen by random sampling. Using random locations reduces
the chance that participants might inducing the location coincidentally.
Locations 1, 7, 10 and 11 were not in the sample.

The lineup locations were the same for both map types, because each set
of lineup data was used to produce a choropleth map lineup and hexagon
tile map lineup. Lineups were grouped into A or B, so that a participant
saw only one version. Participants were assigned to group A or B,
randomly, and thus evaluated either the choropleth or the hexagon tile
map lineup. Because there were four replicates of each lineup, each
participant evaluated two choropleth and two hexagon tile map lineups,
for each trend model. This design is illustrated in
Figure~\ref{fig-exp-design}.

For each of the 144 individual maps, the values for each geographic area
were rescaled to create a similar color scale from deep blue to dark red
within each map. This meant at least one geographic unit was coloured
dark blue, and at least one was red, in every map display of every
lineup.

For the geographic NW-SE distribution, this resulted in the smallest
values of the trend model (blue) occurring in Western Australia, the
North West of Australia, and the largest values of the trend model (red)
occurring in the South East. This resulted in Tasmania being colored
completely red.

For the population related displays, the clusters in the cities appeared
more red than the rest of Australia.

\subsection{Analysis}\label{analysis}

\subsubsection{Data Cleaning}\label{data-cleaning}

The first step in the data cleaning process involved checking that
survey responses collected for each participants were only included once
in the data set. The data cleaning process also involved filtering out
participants' who did not provide at least three unique choices when
considering each of the twelve lineups. These participants achieved a
detection rate of 0. If participants had made various plot choices for
the 12 displays they saw they were still included in the dataset.

\subsubsection{Descriptive statistics}\label{descriptive-statistics}

Basic descriptive statistics were used to contrast the detection rate
for the two types of displays. Comparison was also made across the trend
models, contrasting the mean and standard detection rate for each group,
who had seen the different map display type for each replicate.

Side-by-side dot plots were made of accuracy (efficiency) against map
type, faceted by trend model type.

Similar plots were made of the feedback and demographic variables -
reason for choice, reported difficulty, gender, age, education, having
lived in Australia - against the design variables.

\subsubsection{Modelling}\label{modelling}

The likelihood of detecting the data plot in the lineup can be modelled
using a linear mixed effects model. The R \texttt{glmer()} function in
the \texttt{lme4} \citep{lme4} package implements generalised linear
mixed effect models. The model used includes the two main effects map
type and trend model, which gives the fixed effects model to be:

\[
\hat{y}_{ij} \sim Bernoulli(p_ij)
\] with

\[
\text{logit}(p_{ij}) = \mu + \tau_i + \delta_j + (\tau \delta)_{ij} ~~~~ i=1,2; ~~ j=1,2,3
\]

where \(y_{ij} = 0, 1\) represents whether the subject detected the data
plot (1) or did not (0), \(\mu\) is the overall mean, \(\tau_i, i=1,2\)
is the map type effect, \(\delta_j\) is the trend model effect. We are
allowing for an interaction between map type and trend model as the
response is binary, so a logistic model was used. As each participant
provides results from 12 lineups, this model can account for each
individual participants' abilities as it includes a subject-specific
random intercept.

The model specifies a logit link, this means the predicted values from
the \texttt{glmer} model should be back-transformed to fit between 0 and
1. The predictions \(\widehat{p}(\eta)\) are transformed to be
probabilities between 0 and 1 with the link specified below:

\[\widehat{p}(\eta) = \frac{e^{\eta}}{1 + e^{\eta}}\]
\label{eq:transform} \[\eta = f(\tau_i,\delta_j)\]

\subsection{Web application to collect
responses}\label{web-application-to-collect-responses}

The taipan \citep{taipan} package for R was used to create the survey
web application. This structure was altered to collect responses
regarding participants demographics and their survey responses. The
survey app contained three tabs. Participants were first asked for their
demographics their unique identifier and their consent to the responses
being used for analysis. The demographics collected included
participants' preferred pronoun, the highest level of education
achieved, their age range and whether they had lived in Australia.

After submitting these responses, the survey application switched to the
tab of lineups and associated questions. This allowed participants to
easily move through the twelve displays and provide their choice, reason
for their choice, and level of certainty.

When participants completed the twelve evaluations the survey
application triggered a data analysis script. This created a data set
with one row per evaluation. Containing the responses to the three
questions. The script also added the title of the image, which indicated
the type of map display, the type of distribution hidden in the lineup,
and the location of the data plot. It also calculated the time taken by
participant to view each lineup.

Each participant used the internet to access the survey. The data
transfer from the web application to the data set took place using a
secure link to the googlesheet used to store results. The application
connected to the googlesheet using the googlesheets \citep{sheets} R
package when participants opened the application, and interacted again
when participants chose to submit the survey. At this time it added the
participant's responses to the twelve lineup displays as twelve rows of
data in the googlesheet.

\subsection{Participants}\label{participants}

Participants were recruited from the Figure Eight crowdsourcing platform
\citep{figeight} to evaluate lineups. The lineup protocol expects that
the participants are uninvolved judges with no prior knowledge of the
data, to avoid inadvertently affecting results. Potential participants
needed to have achieved level 2 or level 3 from prior work on the
platform. All participants were at least 18 years old.

Participants were allocated to either group A or group B when they
proceeded to the survey web application. There were 92 participants
involved in the study. All participants read introductory materials, and
were trained using three test displays, to orient them to the evaluation
task. All participants who completed the task were compensated \$AUD5
for their time, via the Figure Eight payment system.

A pilot study was conducted in the working group of the Econometrics and
Business Statistics Department of Monash University. This allowed us to
estimate the effect size, and thus decide on number of participants to
collect responses from.

\subsection{Demographic data
collection}\label{demographic-data-collection}

Each participant answered demographic questions and provided consent
before evaluating the lineups.

Demographics were collected regarding the study participants:

\begin{itemize}
\tightlist
\item
  Gender (female / male / other),
\item
  Education level achieved (high school / bachelors / masters /
  doctorate / other),
\item
  Age range (18-24 / 25-34 / 35-44 / 45-54 / 55+ / other)
\item
  Lived at least for one year in Australia (Yes / No )
\end{itemize}

Participants then moved to the evaluation phase. The set of images
differed for Group A and Group B. After being allocated to a group, each
individual was shown the 12 lineups in randomised order, and asked to
report:

\begin{itemize}
\tightlist
\item
  Plot choice: the number of the plot that they deemed to be most
  different from the others.
\item
  Reason: one of ``Clusters of colour'', ``Colour trend across the
  areas'', ``Big differences between neighbouring areas'', ``All areas
  have similar colours'' or ``None of these reasons''.
\item
  Certainty: how certain that their choice is different from the others,
  on a scale of 1-5.
\end{itemize}

on each of lineup. These responses are saved after entering, but only
recorded in the results sheet when the participants clicked the submit
button on the final page.

\section{Results}\label{sec-results}

Responses from 92 participants were collected. Five participants did not
provide more than three unique choices for the twelve lineups, and their
data was removed. Set A was evaluated by 42 participants, and 53
evaluated set B. This resulted in 1104 evaluations, corresponding to 92
subjects, each evaluating 12 lineups, that were analysed on accuracy and
speed. The certainty and reasons of subjects in their answers is also
examined.

\subsection{Participant demographics}\label{participant-demographics}

Of the 92 participants, 67 were male, and 25 female. Most participants
(56) had a Bachelors degree, 13 had a Masters degree, and the remaining
23 had high school diplomas.

\subsection{Accuracy}\label{accuracy}

Figure~\ref{fig-detect-compare} displays the average detection rates for
the two types of plot separately for each trend model. Each trend model
was tested using four repetitions, evaluations on the same data set were
seen as either choropleths or hexagon tile maps by each group as
specified in Figure~\ref{fig-exp-design}; the detection rates for each
display are connected by a line segment. The Three Cities and All Cities
trend models shown in the hexagon tile map allowed viewers to detect the
data plot substantially more often than the choropleth counterparts. One
replicate for the All Cities group had similar detection rates for both
map types, the rate of detection using the choropleth map was much
higher than other replicates. Surprisingly, participants could also
detect the gradual spatial trend in the NW-SE group from the hexagon
tile map. We expected that the choropleth map would be superior for the
type of spatial pattern, but the data suggests the hexagon tile map
performs slightly better, or equally as well.

\begin{figure}

\centering{

\includegraphics[width=1\linewidth,height=\textheight,keepaspectratio]{kobakian-cook-anzjs_files/figure-pdf/fig-detect-compare-1.pdf}

}

\caption{\label{fig-detect-compare}The detection rates achieved by
participants are contrasted when viewing the four replicates of the
three trend models. Each point shows the probability of detection for
the lineup display, the facets separate the trend models hidden in the
lineup. The points for the same data set shown in a choroleth or hexagon
tile map display are linked to show the difference in the detection
rate.}

\end{figure}%

Table~\ref{tbl-desc-stats} shows the means and standard deviations of
the detection rate for each type of plot and each trend model. This also
gives the standard deviations, the smallest standard deviation for all
sets of replicates was the Three Cities trend model shown in a
Choropleth display. This group of displays had a very small detection
rate of 0.04. The mean detection rate for the Three Cities trend model
shown as choropleth map lineups was also the smallest at 0.40. The
North-West to South-East (NW-SE) trend model unexpectedly had a higher
mean detection rate for the hexagon tile map displays, but the
difference in the means of detection rate was only 0.10.

\begin{table}

\caption{\label{tbl-desc-stats}}

\centering{

\caption{\label{tab:tbl-desc-stats}The mean and standard deviation of the rate of detection for each trend model, calculated for the choropleth and hexagon tile map displays.}
\centering
\begin{tabular}[t]{lccc}
\toprule
Type & NW-SE & Three Cities & All Cities\\
\midrule
Choro. & 0.52 & 0.04 & 0.23\\
 & (0.50) & (0.19) & (0.42)\\
\addlinespace
Hex. & 0.62 & 0.40 & 0.58\\
 & (0.49) & (0.49) & (0.49)\\
\bottomrule
\end{tabular}

}

\end{table}%

Table~\ref{tbl-detect-glmer1} presents a summary of the generalised
linear mixed effects model, testing the effect of map type and trend
model on the detection rate. The results support the summary from
Figure~\ref{fig-detect-compare} and all parameters are statistically
significant despite the large standard deviations observed in
Table~\ref{tbl-desc-stats}. Overall, the hexagon tile map performs
marginally better than the choropleth for all trend models, which is a
pleasant surprise. Allowing for the interaction effect, the difference
in detection rate decreases for population related displays for a
choropleth map lineup, but increases for a hexagon tile map display. The
log odds of detection show in Table~\ref{tbl-detect-glmer1} can be back
transformed after taking the sum of all terms for the trend and type of
display that are of interest. For the NW-SE distribution, the predicted
detection rate for the hexagon tile map display increases the predicted
probability of detection to 0.63 from 0.52 for choropleths, this is
almost exactly the difference seen in the table of means and is
significant only at the 0.05 level.

When a choropleth map display is used, the predicted detection rate for
the Three Cities trend, 0.03; this is extremely low, especially compared
to the NW-SE trend of 0.52. When the All Cities trend is presented in a
choropleth display the predicted probability of detection is 0.22. The
hexagon tile map has a substantially high detection rate for the display
of a Three Cities trend 0.39 and All Cities trend 0.59.

\begin{table}

\caption{\label{tbl-detect-glmer1}}

\centering{

\caption{\label{tab:tbl-detect-glmer1}The model output for the generalised linear mixed effect model for detection rate. This model considers the type of display, the trend model hidden in the data plot, and accounts for contributor performance.}
\centering
\begin{tabular}[t]{rrlrr}
\toprule
Term & Est. & Sig. & Std. Error & P val\\
\midrule
Intercept & -1.27 & $^{***}$ & 0.19 & 0.00\\
Hex. & 1.63 & $^{***}$ & 0.24 & 0.00\\
\addlinespace
Three Cities & -2.07 & $^{***}$ & 0.43 & 0.00\\
All Cities & 1.34 & $^{***}$ & 0.24 & 0.00\\
\addlinespace
Hex:Three Cities & 1.28 & $^{**}$ & 0.48 & 0.01\\
Hex:All Cities & -1.16 & $^{***}$ & 0.33 & 0.00\\
\bottomrule
\end{tabular}

}

\end{table}%

\subsection{Speed}\label{speed}

Figure~\ref{fig-beeswarm} shows horizontally jittered dot plots to
contrast the time taken by participants to evaluate each lineup when
viewing each type of display. The time are also separated by trend model
and whether the data plot was detected or not detected. The time taken
to complete an evaluation ranged from milliseconds to 60 seconds. The
average time taken for type of display is shown as a large colored dot
on each plot. when considering the heights of the green and orange dots,
there is little difference in the average time taken to read a
choropleth or hexagon tile map. Comparing the same colored dot across
each trend model row, there is a slight increase in the time taken to
correctly detected the data plot in the hexagon tile map lineup, but
little difference in evaluation time for the choropleth display.
However, there were substantially less correct detections for choropleth
lineups for the Three cities and All Cities trends.

\begin{figure}

\centering{

\includegraphics[width=1\linewidth,height=\textheight,keepaspectratio]{kobakian-cook-anzjs_files/figure-pdf/fig-beeswarm-1.pdf}

}

\caption{\label{fig-beeswarm}The distribution of the time taken
(seconds) to submit a response for each combination of trend, whether
the data plot was detected, and type of display, shown using
horizontally jittered dotplots. The colored point indicates average time
taken for each map type. Although some participants take just a few
seconds per evaluation, and some take as much as mcuh as 60 seconds, but
there is very little difference in time taken between map types.}

\end{figure}%

\subsection{Certainty}\label{certainty}

Participants provided their level of certainty regarding their choice
using a five point scale. Unlike the accuracy and speed of responses
that were derived during the data processing phase, this was a
subjective assessment by the participant prompted by the question: `How
certain are you about your choice?'. Figure~\ref{fig-certainty} shows
the amount of times participants provided each level of certainty. This
was separated for each combination of trend models and display type, and
colored depending on whether a participant correctly detected the data
plot in the lineup. Participants often chose 4 or 5 when viewing the
population related trends in the choropelth display, even though they
were often incorrect when viewing an All Cities trend and overwhelmingly
incorrect for the Three Cities trend. This shows overconfidence in their
detection ability when using a choropleth map display. Participants were
less likely to be certain when their choice was incorrect and they were
viewing a hexagon tile map. For each trend model, participants were more
likely to doubt their choice and choose 1 or 2 in the hexagon tile map
displays, even though many had made the correct choice.

\begin{figure}

\centering{

\includegraphics[width=1\linewidth,height=\textheight,keepaspectratio]{kobakian-cook-anzjs_files/figure-pdf/fig-certainty-1.pdf}

}

\caption{\label{fig-certainty}The amount of times each level of
certainty was chosen by participants when viewing hexagon tile map or
choropleth displays. Participants were more likely to choose a high
certainty when considering a Choropleth map. The mid value of 3 was the
default certainty, it was chosen most for the Hexagon tile map
displays.}

\end{figure}%

\subsection{Reason}\label{reason}

Participants were asked why they had made their plot choice and were
able to select from a set of suggested reasons. ``Color trend across the
areas'' was the most common selection for NW-SE trend displays
(Table~\ref{tbl-reason}).

The reasons chosen by participants from the list provided to them varied
more when viewing choropleth displays than the hexagon tile map. The
hexagon tile map displays resulted in ``Clusters of color'' as the most
common choice made by participants.

The choice ``None of these reasons'' was used as the default value to
minimise noise from participants who did not select a response.

\begin{table}

\caption{\label{tbl-reason}}

\centering{

\caption{\label{tab:tbl-reason}The amount of participants that selected each reason for their choice of plot when looking at each trend model shown in Choropleth and Hexagon Tile maps. The facets show whether or not the choice was correct.}
\centering
\begin{tabular}[t]{llll}
\toprule
Trend & Detected & Choro. & Hex.\\
\midrule
 & No & trend & clusters\\
\cmidrule{2-4}
\multirow{-2}{*}{\raggedright\arraybackslash All Cities} & Yes & clusters, consistent & clusters\\
\cmidrule{1-4}
 & No & trend & clusters\\
\cmidrule{2-4}
\multirow{-2}{*}{\raggedright\arraybackslash Three Cities} & Yes & consistent & clusters\\
\cmidrule{1-4}
 & No & trend & clusters\\
\cmidrule{2-4}
\multirow{-2}{*}{\raggedright\arraybackslash NW-SE} & Yes & trend & clusters\\
\bottomrule
\end{tabular}

}

\end{table}%

\section{Discussion}\label{discussion}

The intention of this study was to contrast the use of the choropleth
map and the hexagon tile map. The visual inference lineup protocol was
employed to contrast the effectiveness of the displays. The results have
shown that overall the use of the hexagon tile map display allows
participants to find the data plot in the lineup more often. Using the
visual inference protocol this result can be extended to show that it is
a valid alternative display to communicate spatial distributions of
population related data.

We expected that the choropleth map would be superior for communicating
the spatial pattern of geographic distributions. The data suggest that
the participants perform slightly better or equally as well for each
replicate in each trend model across the two displays. Table II shows
that the difference in the mean detection rate for the two trend models
was 0.10.

The differences seen in the Figure~\ref{fig-detect-compare} plot and
Table~\ref{tbl-desc-stats} are reflected in the model results.
Surprisingly the difference for the geographic distribution was
significant at the 0.05 level. It also showed that the hexagon tile map
display performs marginally better than the choropleth for all trend
models. Unexpectedly the detection rate suffers when using a choropleth
map to display population related distributions.

While the significance of the difference in detection was the key focus
of this experiment, the secondary focus was the time taken by
participants. it was expected that the participants may take longer to
consider the hexagon tile map distribution but would be able to detect
the data plot in the lineup. The bimodal distributions seen in
Figure~\ref{fig-beeswarm} showed very little difference in the mean
evaluation times. As the maximum time of all of the distributions
approached 60 seconds it cannot be said that the participants' took
longer to evaluate the hexagon tile map displays.

The responses to the questions asked of participants included the reason
for their choice and the certainty around their choice.
Table~\ref{tbl-reason} shows high levels of certainty of 4 and 5 were
chosen by participants when looking at the population distributions in a
choropleth map display show that they were over confident when
attempting to find the real data plot in the choropleth map displays.
Participants performed better on the NW-SE distribution shown in the
choropleth display and were reasonably confident about their decisions.
The high levels of the mid range value of 3 could indicate that the
participant did not want to provide a response, as this was the default
value. Those who chose level 4 or 5 were equally likely to be correct
for the three cities lineups, but more likely to be correct than
incorrect for the other two trend models.

The color scaling applied in Three cities and All cities displays
resulted in the rural areas of the real data plot appearing more blue or
yellow than the other plots in the lineups. Due to the consistent
coloring of rural areas in a choropleth display, the choice ``All areas
have similar colors'' was most common reason for a participants choice.
The All Cities displays colored the inner-city areas of all capital
cities more red, this was observable to participants and explains the
equal choice of the city clusters or rural color consistency. Choosing
``Clusters of colour'' was expected when participants viewed the Hexagon
tile map display of the All Cities and Three Cities distributions. It
was unexpected that it was also the most common reason for the NW-SE
hexagon tile map displays. Due to the spatial covariance introduced in
the smoothing, groups of similarly colored hexagons were present in all
of the hexagon tile map displays. All Cities and Three Cities
distributions of real data trends had distinctly different patterns or
red inner-city areas, while some of the plots in each lineup may have
shared similar features.

\section{Conclusion}\label{conclusion}

The choropleth map display and the tessellated hexagon tile map have
been contrasted using the lineup protocol. The hexagon tile map was
significantly more effective for spotting a real population related data
trend model hidden in a lineup.

The hexagon tile map display should be considered as an alternative
visualization method when communicating distributions that relate to the
population across a set of geographic units. As an additional display to
the familiar choropleth map, cancer atlas products may benefit from the
opportunity to allow exploration via an alternative display. The spatial
distributions used to test these displays were inspired by the real
spatially smoothed estimates of the cancer burden on Australian
communities. However, this technique may be extended to other population
related distributions, such as other diseases.

The increasing population densities of capital cities despite large land
area exacerbates the difference in the smallest and largest communities.
The population density structure of Australia can be considered similar
to that of Canada, New Zealand and many other countries. Therefore, this
display is not only relevant to Australia, but all nations or population
distributions that experience densely populated cities separated by vast
rural expanses.

\section*{Acknowledgments}\label{acknowledgments}
\addcontentsline{toc}{section}{Acknowledgments}

The authors would like to thank the Australian Cancer Atlas team for
discussions regarding alternative spatial visualizations, and Professor
Kerrie Mengersen and Dr Earl Duncan for regular meetings filled with
suggestions and comments. Mitchell O'Hara-Wild was a co-developer of the
taipan \citep{taipan} R package for image tagging, used as the base for
the web app constructed to collect participant evaluations of lineups.
We are thankful for the NUMBATs (Non-Uniform Monash Business Analytics
Team) for participating in the pilot study that helped to assess the
experimental design and determine an appropriate sample size for the
study.

The source code to produce this document can be found on
\href{https://github.com/srkobakian/experiment/paper}{GitHub}.
Supplementary materials have been included to discuss the survey
procedures and the lineups that were used. The full set of images can be
found here, too.

The supplementary material contains:

\begin{itemize}
\tightlist
\item
  Additional analysis of the experimental results
\item
  Survey procedure including training materials for the participants
\item
  24 lineups as images, that were used in the experiment
\item
  12 data sets used to construct the lineups
\end{itemize}

The analysis of the work was completed in R \citep{RCore} with the use
of the following packages:

\begin{itemize}
\tightlist
\item
  Document creation: quarto \citep{Allaire_Quarto_2025}, anzjs template
  \citep{quarto-anzjs}, knitr \citep{knitr}.
\item
  Lineup creation: nullabor \citep{nullabor}, gstat \citep{gstat}.
\item
  Data analysis: tidyverse \citep{tidyverse}, ggthemes \citep{ggthemes},
  RColorBrewer \citep{RColorBrewer}.
\item
  Plots: ggplot2 \citep{ggplot2}, cowplot \citep{cowplot}, png
  \citep{png}, grid \citep{grid}.
\item
  Modelling and summary presentation: lme4 \citep{lme4}, kableExtra
  \citep{kableExtra}.
\end{itemize}

Ethics approval for the online survey was granted by QUT's Ethics
Committee (Ethics Application Number: 1900000991). All applicants
provided informed consent in line with QUT regulations prior to
participating in this research.


\bibliography{paper.bib}



\end{document}
